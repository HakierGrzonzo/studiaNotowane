
% !TEX program = xelatex
\documentclass[12pt,aspectratio=169]{beamer}
\usetheme{Warsaw}
\usepackage[polish]{babel}
\usepackage{tikz}
	\usetikzlibrary{arrows}
	\usetikzlibrary{patterns}

\usepackage{polski}
\usepackage{lmodern}
\usepackage{graphicx}
\usepackage[backend=bibtex]{biblatex}
\usepackage{csquotes}

\title{API4:2023 Unrestricted Resource Consumption}
\author{Grzegorz Koperwas}
\addbibresource{bib.bib}

\begin{document}

\begin{frame}
\maketitle
\end{frame}

\section{Wprowadzenie}
\begin{frame}
W sprawozdaniu omówimy sposoby przeciwdziałania podatności \texttt{API4:2023
Unrestricted Resource Consumption}. Będziemy omawiać je w kontekście aplikacji 
wykorzystujących rozwiązania oparte na LLM.
\end{frame}

\subsection{Założenia dot. zadania laboratoryjnego}

\begin{frame}
LLM, z racji ogromnej ilości parametrów, wymagają dużych ilości pamięci. Są
również nie przewidywalne, podatne na ataki \emph{prompt injection} oraz są
nowym trendem, przez co nie koniecznie mamy zawsze do czynienia z przemyślanym
rozwiązaniem.
\end{frame}

\begin{frame}
Z tego powodu będziemy się skupiać na rozwiązaniach, które nie tylko ograniczają
prawdopodobieństwo zablokowania się systemu, ale również ograniczających jego
podatności na ataki \emph{prompt injection}.

Będziemy ewaluowali rozwiązania w ramach ich przydatności do zabezpieczenia 
przykładowej aplikacji realizującej \emph{Retrieval Augmented Generation}, w
skrócie \texttt{RAG}. 
\end{frame}

\begin{frame}
Aplikacje te wykorzystują wektorowe bazy danych, gdzie przechowywane są dane. W
ramach zapytania baza taka potrafi dobrać informacje, które pomagają
odpowiedzieć \texttt{LLM} na zadane pytania. Korzystają one z wyspecjalizowanych
sieci zwanych \emph{embeddings}, które zamieniają tekst na wektor. Mając nasz
tekst w formie wektorowej, możemy łatwo porównywać dystans między różnymi
elementami, czyli pośrednio ich podobieństwo.
\end{frame}

\section{Rozwinięcie}
\begin{frame}
Podstawową linią obrony przeciwko zbyt dużemu zużyciu zasobów jest ograniczenie
tego, ile pojedyncze zapytanie do systemu może wywoływać modele \texttt{LLM}.
Innym czynnikiem, który musimy kontrolować, jest \emph{kontekst} oraz jego
rozmiar. Modele \texttt{LLM} operują nie na nieskończonej ilości danych, tylko
na pewnym wycinku tekstu, jaki jest im zadany jako argument.
\end{frame}

\begin{frame}
Te przysłowiowe okienko, przez które na problem spoglądają \texttt{LLM} nazywamy 
kontekstem, nic poza nim nie ma wpływu na sieć. Jego rozmiar jest mierzony w
\emph{tokenach}, które mniej-więcej odpowiadają słowom lub częściom słów. Każdy
model ma określony rozmiar kontekstu, przykładowo dla modeli \texttt{LLama} mamy
do czynienia z kontekstem wielkości 2048 tokenów, a niektóre z modeli
\texttt{GPT} mają konteksty rozmiaru nawet 128 tysięcy tokenów, dla modelu
\texttt{GPT-4 Turbo}.
\end{frame}

\begin{frame}
Jednak musimy pamiętać, że często koszt użycia modelu jest wyznaczany na
podstawie ilości tokenów, które zostają przekazana jako kontekst, jak i ilości
wygenerowanych tokenów.
\end{frame}

\subsection{Kontrola rozmiaru kontekstu}
\begin{frame}
Podczas budowania naszej aplikacji wykorzystującej \texttt{RAG} możemy chcieć
dać modelowi dostęp do całej naszej bazy danych. Jednak przekazanie w kontekście
wszystkiego jak leci, nie zadziała z następujących powodów:

\begin{itemize}
  \item Będzie to kosztowne, ponieważ płacimy za każdy token.
  \item Będzie to nieefektywne, ponieważ nasze dane nie zmieszczą się w
    kontekście.
\end{itemize}
\end{frame}
\begin{frame}
W celu rozwiązania tego problemu możemy skorzystać z biblioteki
\emph{langchain}, która pomaga nam zarządzać kontekstem. 

Możemy skorzystać z jej pomocy w celu integracji z wektorową bazą danych, na
przykład \emph{chromadb}. Pozwoli nam to do naszego kontekstu wprowadzać tylko
dokumenty (lub ich fragmenty) faktycznie związane z naszym problemem.

Z użyciem takiej bazy wektorowej możemy ograniczyć przetwarzane informacje tylko
do konkretnych dokumentów lub ich fragmentów, co pozwala nam używać mniejszych
kontekstów, co ogranicza koszty pojedynczego zapytania.
\end{frame}
\begin{frame}
Dodatkowo możemy zastosować prosty limit ilości tokenów, które może model
wygenerować. Inną opcją kontroli wyników modeli jest, na przykład w narzędziu
\texttt{llama.cpp}, jest zadawanie gramatyki, jaką musi spełniać odpowiedź.

Przykładowo, zamiast ograniczać ilość tokenów, możemy modelowi uniemożliwić
wygenerowanie więcej niż dwóch zmian. Możemy również nakazać modelowi
generowanie odpowiedzi w konkretnym formacie, na przykład \texttt{yaml}, zamiast
opisywać mu w kontekście, jakie warunki ma spełniać jego odpowiedź.
\end{frame}

\subsection{Kontrola zapytań}
\begin{frame}
Dodatkowo możemy stosować tradycyjne metody kontroli ilości zapytań, na
przykład:

\begin{itemize}
  \item Ograniczenia ilości zapytań dla klienta.
  \item Ograniczenia rozmiaru zapytania.
\end{itemize}

Jednak te metody nie są specyficzne dla modeli \texttt{LLM}, więc nie chcę ich
szczegółowo omawiać w tym referacie.
\end{frame}

\newpage

\section{Podsumowanie i wnioski}
\begin{frame}
\begin{itemize}
  \item \textit{Podsumowanie} - Modele \texttt{LLM} mogą korzystać zarówno z
    tradycyjnych metod zabezpieczania \texttt{API}, jak i z innych technik
    pozwalających optymalizować ich działanie.
  \item \textit{Wnioski} - Koszt wykorzystania modeli \texttt{LLM} jest zależny
    od rozmiaru kontekstu, dlatego musimy kontrolować jego rozmiar. Jest to
    podobne do tego, czemu stosujemy paginację.
\end{itemize}
\end{frame}

\end{document}
