% !TEX program = xelatex
%Wzór dokumentu
%tu zmień marginesy i rozmiar czcionki
\documentclass[a4paper,12pt]{article}
\usepackage{inputenc}[utf8]
\usepackage[margin=1.5cm]{geometry}
\usepackage[polish]{babel}

%Lepiej tego nie zmieniaj, jak co to dodawaj pakiety
\usepackage{titlesec}
\usepackage{titling}
\usepackage{fancyhdr}
\usepackage{mdframed}
\usepackage{graphicx}
\usepackage{amsmath}
\usepackage{amsfonts}
\usepackage{multicol}
\usepackage{multirow}
\usepackage{listings}
\usepackage{caption}
\usepackage{float}
\usepackage{pdfpages}
\usepackage{tikz}
	\usetikzlibrary{arrows}
	\usetikzlibrary{patterns}
	\usetikzlibrary{decorations.pathmorphing}
\usepackage{pgf}
\usepackage[section]{placeins}

\usepackage{pict2e}

\makeatletter
\newcommand{\odpowiada}{\mathrel{\mathpalette\raising@circles@eq\relax}}
\newcommand{\raising@circles@eq}[2]{%
  \vphantom{#1+}%
  \vbox{
    \settowidth\unitlength{$#1\mspace{2mu}$}%
    \offinterlineskip\m@th
    \ialign{##\cr
      \hfil\small@circle{#1}$#1\mspace{1.5mu}$\cr\noalign{\vskip0.5\unitlength}
      $#1=$\cr\noalign{\post@vskip{+}{#1}}
      $#1\mspace{1.5mu}$\small@circle{#1}\hfill\cr\noalign{\post@vskip{-}{#1}}
    }%
  }%
}
\newcommand{\small@circle}[1]{%
  \smash{%
    \begin{picture}(1,1)
    \small@linethickness{#1}
    \put(0.5,0.5){\circle{1}}
    \end{picture}%
  }%
}
\newcommand{\small@linethickness}[1]{%
  \linethickness{%
      \ifx#1\displaystyle 0.8\fontdimen8\textfont3\else
      \ifx#1\textstyle 0.8\fontdimen8\textfont3\else
      \ifx#1\scriptstyle0.8\fontdimen8\scriptfont3\else
      1\fontdimen8\scriptscriptfont3\fi\fi\fi
  }%
}
\newcommand{\post@vskip}[2]{%
  \expandafter\vskip\expanded{%
    #1\ifx#2\scriptscriptstyle0.9\else\ifx#2\scriptstyle0.6\else0.3\fi\fi\unitlength
  }%
}
\makeatother


%inny wygląd
%\usepackage{tgbonum}


\usepackage{hyperref}
\hypersetup{
    colorlinks=true,
    linkcolor=blue,
    filecolor=magenta,      
    urlcolor=cyan,
}

\urlstyle{same}
%Zmienne, zmień je!
\graphicspath{ {./ilustracje/} }
\title{Ściąga na matematykę dyskretną}
\author{Grzegorz Koperwas}
\date{\today}

%lokalizacja polska (odkomentuj jak piszesz po polsku)

\usepackage{polski}
\usepackage[polish]{babel} 
\usepackage{indentfirst}
\usepackage{icomma} 

\brokenpenalty=1000
\clubpenalty=1000
\widowpenalty=1000    

%nie odkometowuj wszystkiego, użyj mózgu
%\renewcommand\thechapter{\arabic{chapter}.}
\renewcommand\thesection{\arabic{section}.}
\renewcommand\thesubsection{\arabic{section}.\arabic{subsection}.}
\renewcommand\thesubsubsection{\arabic{subsubsection}.}

%Makra

\newcommand{\obrazek}[2]{
\begin{figure}[h]
    \centering
    \includegraphics[scale=#1]{#2}
\end{figure}
}     

\newcommand{\stopnie}{\ensuremath{^{\circ}}}

\newcommand{\twierdzonko}[1]{
    \begin{center}
    \begin{mdframed}
    #1
    \end{mdframed}          
    \end{center}
} 

\newcommand{\dwanajeden}[2]{
\ensuremath \left( \begin{array}{c}
    #1\\
    #2
\end{array} \right)
}  

%Stopka i head (sekcja której nie powinno się zmieniać)
\pagestyle{fancy}
\fancyhead{}
\fancyfoot{}

%Zmieniaj od tego miejsca
\rfoot{\thepage}
\lfoot{}
\lhead{}
\rhead{Ostatnia edycja: \today}
\renewcommand{\headrulewidth}{1pt}
\renewcommand{\footrulewidth}{1pt}



\begin{document}

\begin{multicols}{2}
    
    \section{Modulo}

    \[
        a \equiv b \mod m \Leftrightarrow m | a - b
    \]

    \section{Chińskie reszty}

    Niech $m_1, \dots, m_n$ to liczby parami względnie pierwsze. $r_i \in Z$.
    \[
        NWD\left(m_i, m_{i+1}\right) = 1
    \]
    
    Wtedy równanie:

    \begin{equation*}
        \left\{ 
        \begin{matrix}
            x \equiv_{m_1} r_1 \\
            x \equiv_{m_2} r_2 \\ 
            x \equiv_{m_n} r_n \\ 
        \end{matrix}
        \right.
    \end{equation*}

    Ma jedno rozwiązanie $\mod M = {\prod_{i=1}^{n} m_i}$ postaci:
    \begin{align*}
        x &= N_1 M_1 + \cdots + N_n M_n,\quad \text{gdzie:} \\
        M_i &= \frac{M}{m_i} \\
        M_i N_i &\equiv_{m_i} r_i \\
    \end{align*}




    \section{Algosy}

    \subsection{Euklides NWD}
    \[
        a, b \in N, a > b 
    \]
    Dzielenie z resztą $a / b$, jeżeli reszta $r \not= 0$ to dzielimy z resztą $b /
    r_1$. Ostatnia nie zerowa reszta to wynik.

    \subsubsection{Równania liniowe}

    \[
        NWD\left(a, b, c\right) = NWD\left(NWD\left(a, b\right), c\right)
    \]

    \[
        ax + by = c \Leftrightarrow NWD\left(a, b\right) | c
    \]
    \[
        ax = b \text{ w } Z_m \Leftrightarrow NWD\left(a, m\right) | b
    \]
    \[ 
        x_n = x_0 + \frac{bn}{NWD\left( a, b \right)}\quad
        y_n = y_0 - \frac{an}{NWD\left( a, b \right)}
    \]







\end{multicols}
\end{document}
