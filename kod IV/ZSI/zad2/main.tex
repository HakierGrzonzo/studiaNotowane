\documentclass[12pt,a4paper]{article}

\usepackage[T1]{fontenc}
\usepackage{polski}
\usepackage[polish]{babel}
\usepackage[utf8]{inputenc}
\usepackage{lmodern}
\selectlanguage{polish}
\usepackage{graphicx}

\begin{document}
\pagenumbering{gobble}
\clearpage
\begin{figure}[h]
\centering
\includegraphics{media/ps-logo.png}
\end{figure}
\hspace{3cm}
\begin{center}Dokumentacja realizowanych zadań\end{center}
\begin{center}2021/2022\end{center}
\hspace{3cm}
\begin{center}\large\textbf{Zarządzanie systemami informatycznymi}\end{center}
\begin{center}\large\textit{Temat 2}\end{center}

\hspace{7cm}
\begin{flushright}Kierunek: Informatyka
\end{flushright}
\begin{flushright}Członkowie zespołu:
\par
\textit{Grzegorz Koperwas}
\end{flushright}
\vfill
\begin{center}Gliwice, 2021/2022\end{center}

\newpage
\pagenumbering{arabic}
\tableofcontents

\newpage
\section{Wprowadzenie}

\subsection{Role w realizacji zadania}
\ldots 

\subsection{Cel realizacji zadania}
\ldots 

\newpage

\section{Założenia realizacji zadania}

\subsection{Założenia techniczne i nietechniczne}
\ldots 

\subsection{Stos technologiczny}
Narzędzia i systemy informatyczne związne z projektem

\subsection{Oczekiwane rezultaty realizacji zadania}
\ldots 

\newpage
\section{Realizacja zadania}
%Opis wszystkich etapów realizacji projektu

\subsection{System operacyjny: konfiguracja i administracja}

\begin{enumerate}
    \item Identyfikacja najważniejszych zadań administracyjnych, które mogą być
        realizowane w przedsiębiorstwie z wykorzystaniem ustawień systemu
        operacyjnego.  \begin{enumerate}
            \item Zarządzanie polityką haseł dla użytkowników
            \item Zarządzanie uprawnieniami do zasobów współdzielonych
            \item Konfiguracja systemu zgodna z procedurami ochrony sekretów 
                przedsiębiorstwa.
            \item Konfiguracja procedur kopii zapasowej.
    \end{enumerate}
    \item Omówienie trzech przykładowych zadań administracyjnych, które mogą
        być realizowane z wiersza poleceń (CMD) \begin{enumerate}
            \item Ustawianie w sposób zautomatyzowany domyślnych aplikacji dla 
                dannych typów plików. (\texttt{assoc})
            \item Resetowanie komputerów o określonym czasie z informacją dla 
                użytkownika (\texttt{shutdown}).
            \item Zarządzanie sterownikami użądzeń (\texttt{Driverquery})
    \end{enumerate}
    \item Omówienie trzech przykładowych poleceń, które pozwalają na realizację
        zadań administracyjnych z wykorzystaniem PowerShell. \begin{enumerate}
            \item Zmiana nazw plików (\texttt{Rename-item})
            \item Przenoszenie plików (\texttt{Move-item})
            \item Uruchamianie serwisów (\texttt{Start-service})
    \end{enumerate}
    \item Pokazanie na trzech przykładach, w jaki sposób narzędzia firm
        trzecich (np. WSCC) pozwalają na realizację zadań administracyjnych.
        \begin{enumerate}
            \item \texttt{WSCC} - Zarządzanie oprogramowaniem na komputerach.
            \item \texttt{samba} - Alternatywny kontroler domeny w Active
                Directory
            \item \texttt{Novel Netware} - Alternatywny system zarządzania
                zasobami w sieci komputerowej, wywodzący się jescze z czasów
                \texttt{MS/DOS}, ale posiadający klienta dla systemu
                \texttt{Windows 10}
        \end{enumerate}
\end{enumerate}

\subsection{Sieci komputerowe: zadania administracyjne}

\begin{enumerate}
    \item Przedstawienie katalogu przydatnych poleceń sieciowych dostępnych w
        wierszu poleceń Microsoft Windows. \begin{enumerate}
            \item \texttt{ping} - wysyłanie \texttt{ICMP ECHO\_REQUEST}.
            \item \texttt{tracert} - wysyłanie \texttt{ICMP ECHO\_REQUEST} z
                małymi \texttt{TTL} w celu mapowania trasy.
            \item \texttt{NetStat} - Statystyki \texttt{tcp/ip}
            \item \texttt{ipconfig} - Wyświetlanie konfiguracji \texttt{ip} oraz
                \texttt{DNS}
        \end{enumerate}
    \item Omówienie procesu diagnostyki połączenia sieciowego (w ramach
        systemu operacyjnego). \begin{enumerate}
            \item \texttt{ipconfig} - sprawdzamy czy komputer ma adres ip, kabel
                ethernet (w przypadku sieci przewodowej) lub zainstalowaną kartę 
                sieciową. 
            \item \texttt{ping/tracert} - sprawdzamy w którym miejscu sieci
                występuje problem.
            \item \texttt{route} - naprawiamy trasy tak by działały.
        \end{enumerate}
    \item Zaproponowanie zestawu trzech narzędzi (dostępnych przez Internet),
        przydatnych w realizacji zadań związanych z administracją sieciami 
        komputerowymi. \begin{enumerate}
            \item \texttt{/www.isitdownrightnow.com} - Czy problem jest po
                naszej stronie.
            \item \texttt{www.whatismyip.com} - Sprawdzanie naszego zewnętrzengo
                adresu \texttt{ip}.
            \item \texttt{ipv6-test.com} - Sprawdzanie czy mamy poprawnie
                skonfigurowane \texttt{ipv6} w sieci.
        \end{enumerate}
\end{enumerate}

\newpage
\section{Wnioski}

\begin{itemize}
\item \textit{Spostrzeżenia}
\item \textit{Osiągnięcia}
\item \textit{Potencjał rozwoju}
\end{itemize}
\end{document}
