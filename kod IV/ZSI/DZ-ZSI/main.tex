\documentclass[12pt,a4paper]{article}

\usepackage[T1]{fontenc}
\usepackage{polski}
\usepackage[polish]{babel}
\usepackage[utf8]{inputenc}
\usepackage{lmodern}
\selectlanguage{polish}
\usepackage{graphicx}

\begin{document}
\pagenumbering{gobble}
\clearpage
\begin{figure}[h]
\centering
\includegraphics{media/ps-logo.png}
\end{figure}
\hspace{3cm}
\begin{center}Dokumentacja realizowanych zadań\end{center}
\begin{center}2020/2021\end{center}
\hspace{3cm}
\begin{center}\large\textbf{Zarządzanie systemami informatycznymi}\end{center}
\begin{center}\large\textit{Temat n}\end{center}

\hspace{7cm}
\begin{flushright}Kierunek: Informatyka
\end{flushright}
\begin{flushright}Członkowie zespołu:
\par
\textit{Adrian Kapczyński}
\par
\textit{Radosław Kapczyński}
\end{flushright}
\vfill
\begin{center}Gliwice, 2020/2021\end{center}

\newpage
\pagenumbering{arabic}
\tableofcontents

\newpage
\section{Wprowadzenie}

\subsection{Role w realizacji zadania}
\ldots 

\subsection{Cel realizacji zadania}
\ldots 

\newpage

\section{Założenia realizacji zadania}

\subsection{Założenia techniczne i nietechniczne}
\ldots 

\subsection{Stos technologiczny}
Narzędzia i systemy informatyczne związne z projektem

\subsection{Oczekiwane rezultaty realizacji zadania}
\ldots 

\newpage
\section{Realizacja zadania}
Opis wszystkich etapów realizacji projektu

\newpage
\section{Wnioski}

\begin{itemize}
\item \textit{Spostrzeżenia}
\item \textit{Osiągnięcia}
\item \textit{Potencjał rozwoju}
\end{itemize}
\end{document}
