% !TEX program = xelatex
%Wzór dokumentu
%tu zmień marginesy i rozmiar czcionki
    \documentclass[a4paper,12pt]{article}
	\usepackage{inputenc}[utf8]
    \usepackage[margin=2.5cm]{geometry}
    
 %Lepiej tego nie zmieniaj, jak co to dodawaj pakiety
	\usepackage{titlesec}
	\usepackage{titling}
	\usepackage{fancyhdr}
	\usepackage{mdframed}
	\usepackage{graphicx}
	\usepackage{amsmath}
	\usepackage{amsfonts}
	\usepackage{multicol}
	
%inny wygląd
	%\usepackage{tgbonum}
	
	
	%Zmienne, zmień je!
	\graphicspath{ {./ilustracje/} }
    \title{Ćwiczenia z Ekonomi}
    \author{Grzegorz Koperwas}
    \date{\today}
    
  %lokalizacja polska (odkomentuj jak piszesz po polsku)
  
    \usepackage{polski}
    \usepackage[polish]{babel} 
    \usepackage{indentfirst}
	\usepackage{icomma} 
	
    \brokenpenalty=1000
    \clubpenalty=1000
    \widowpenalty=1000    
 
 %nie odkometowuj wszystkiego, użyj mózgu
    %\renewcommand\thechapter{\arabic{chapter}.}
	\renewcommand\thesection{\arabic{section}.}
	\renewcommand\thesubsection{\arabic{section}.\arabic{subsection}.}
	\renewcommand\thesubsubsection{\arabic{section}.\arabic{subsection}.\arabic{subsubsection}.}

%Makra
    
\newcommand{\obrazek}[2]{
	\begin{figure}[h]
		\centering
		\includegraphics[scale=#1]{#2}
	\end{figure}
}     
            
    
    \newcommand{\twierdzonko}[1]{
        \begin{center}
        \begin{mdframed}
        #1
        \end{mdframed}          
        \end{center}
    } 
    
    \newcommand{\dwanajeden}[2]{
	\ensuremath \left( \begin{array}{c}
		#1\\
		#2
	\end{array} \right)
}  
      
%Stopka i head (sekcja której nie powinno się zmieniać)
    \pagestyle{fancy}
    \fancyhead{}
    \fancyfoot{}
    \renewcommand{\headrulewidth}{0pt}
    \renewcommand{\footrulewidth}{1pt}
    \rfoot{\theauthor{}}
    
\begin{document}

\begin{titlepage}
    Politechnika Śląska

    Wydział Matematyki Stosowanej

    Kierunek Informatyka


    \begin{center}
        Gliwice, \today

        \vspace{2cm}

        \Large{Programowanie I}

        \vspace{1cm}

        \Large{\bfseries{Założenia do projektu zaliczeniowego}}

        \vspace{1cm}

        \large{,,BrainFuckCompiler64''}

        \vspace{2cm}

        \theauthor{} gr. 2 lab 2C
	\end{center}
	\thispagestyle{empty}
\end{titlepage}

\section{Opis projektu:}
Celem projektu jest stworzenie kompilatora ezoterycznego języka \emph{brainfuck} na procesor \emph{MOS 6502/6510}, dokładnie założone jest skupienie się na komputerze \emph{Commodore 64}.

Program ma akceptować pliki tekstowe z kodem źródłowym i generować z nich pliki assemblera\footnote{Nie będe pisał własnego assemblera, wykorzystany zostanie \emph{kickassembler}}. Pliki po przetworzeniu przez assembler powinny być uruchamiane bez problemów przez emulator \emph{VICE}.

\section{Funkcjonalności:}
\begin{itemize}
    \item Kompilacja do assemblera kickassembler
    \item ośmiobitowa ,,taśma'' o długości większej niż 256 bajtów, w formie \emph{non-wrapping}
    \item Optymailzowanie kodu w postaci \texttt{++++++--+} na pojedyńcze symbole
\end{itemize}



\end{document}