%Wzór dokumentu
%tu zmień marginesy i rozmiar czcionki
    \documentclass[a4paper,12pt]{article}
    \usepackage[margin=3.5cm]{geometry}
    \usepackage{inputenc}[utf8]
    
 %Lepiej tego nie zmieniaj, jak co to dodawaj pakiety
	\usepackage{titlesec}
	\usepackage{titling}
	\usepackage{fancyhdr}
	\usepackage{mdframed}
	\usepackage{graphicx}
	\usepackage{amsmath}
	\usepackage{amsfonts}
	
%inny wygląd
	%\usepackage{tgbonum}
	
	
	%Zmienne, zmień je!
	\graphicspath{ {./ilustracje/} }
    \title{Kolokwium}
    \author{Grzegorz Koperwas}
    \date{\today}
    
  %lokalizacja polska (odkomentuj jak piszesz po polsku)
  
    \usepackage{polski}
    \usepackage[polish]{babel} 
    \usepackage{indentfirst}
	\usepackage{icomma} 
	
    \brokenpenalty=1000
    \clubpenalty=1000
    \widowpenalty=1000    
 
 %nie odkometowuj wszystkiego, użyj mózgu
    %\renewcommand\thechapter{\arabic{chapter}.}
	\renewcommand\thesection{\arabic{section}.}
	\renewcommand\thesubsection{\arabic{section}.\arabic{subsection}.}
	\renewcommand\thesubsubsection{\arabic{section}.\arabic{subsection}.\arabic{subsubsection}.}

%Makra
    
\newcommand{\obrazek}[2]{
	\begin{figure}[h]
		\centering
		\includegraphics[scale=#1]{#2}
	\end{figure}
}     
            
    
    \newcommand{\twierdzonko}[1]{
        \begin{center}
        \begin{mdframed}
        #1
        \end{mdframed}          
        \end{center}
    } 
    
    \newcommand{\dwanajeden}[2]{
	\ensuremath \left( \begin{array}{c}
		#1\\
		#2
	\end{array} \right)
}  
      
%Stopka i head (sekcja której nie powinno się zmieniać)
    \pagestyle{fancy}
    \fancyhead{}
    \fancyfoot{}
    
    %Zmieniaj od tego miejsca
	\rfoot{\thepage}
	\lfoot{Grzegorz Koperwas, \LaTeX note}
	\renewcommand{\headrulewidth}{0pt}
	\renewcommand{\footrulewidth}{1pt}

    
\begin{document}
\maketitle

Grupa C, Zestaw 2

\section{}
\begin{align*}
    \frac{120 \cdot 10^{-6} \cdot 10^{-3} \cdot 16 \cdot 10^{-3}}{\left(25 \cdot 10^{-3}\right)^2} &= \frac{1920}{625} \cdot 10^{-6} \\
    \frac{384}{125} \cdot \frac{10^{-6}}{10^6} &= \frac{384}{125} \cdot 10^{-12} \text{MN}
\end{align*}
\section{}
\begin{align*}
    \left[F \right] &= \frac{kg \cdot m}{s^2} \\
    \frac{kg \cdot m}{s^2} \cdot \frac{m^2}{kg \cdot kg} &= \left[G\right] \\
    \left[G\right] &= \frac{m^3}{kg \cdot s^2}
\end{align*}
\section{}
\begin{align*}
    \vec{a} &= \left[4; 2; -3\right]\\
    \vec{b} &= \left[-9; -5; 6\right]\\
    - \left(\vec{b} \times \vec{a}\right) &= \vec{a} \times \vec{b} = \\
    &= \left[2 \cdot 6 - 3 \cdot 5; 3 \cdot 9 - 4 \cdot 6; 2 \cdot 9 - 4 \cdot 5\right] = \\
    &= \left[-3; 3; -2\right]
\end{align*}
\end{document}
