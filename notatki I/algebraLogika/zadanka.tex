%Wzór dokumentu
%tu zmień marginesy i rozmiar czcionki
    \documentclass[a4paper,12pt]{article}
    \usepackage[margin=3.5cm]{geometry}
	\usepackage{inputenc}[utf8]
    
 %Lepiej tego nie zmieniaj, jak co to dodawaj pakiety
	\usepackage{titlesec}
	\usepackage{titling}
	\usepackage{fancyhdr}
	\usepackage{mdframed}
	\usepackage{graphicx}
	\usepackage{amsmath}
	\usepackage{amsfonts}
	
%inny wygląd
	%\usepackage{tgbonum}
	
	
	%Zmienne, zmień je!
	\graphicspath{ {./ilustracje/} }
    \title{title}
    \author{Grzegorz Koperwas}
    \date{[data]}
    
  %lokalizacja polska (odkomentuj jak piszesz po polsku)
  
    \usepackage{polski}
    \usepackage[polish]{babel} 
    \usepackage{indentfirst}
	\usepackage{icomma} 
	
    \brokenpenalty=1000
    \clubpenalty=1000
    \widowpenalty=1000    
 
 %nie odkometowuj wszystkiego, użyj mózgu
    %\renewcommand\thechapter{\arabic{chapter}.}
	\renewcommand\thesection{\arabic{section}.}
	\renewcommand\thesubsection{\arabic{section}.\arabic{subsection}.}
	\renewcommand\thesubsubsection{\arabic{section}.\arabic{subsection}.\arabic{subsubsection}.}

%Makra
    
\newcommand{\obrazek}[2]{
	\begin{figure}[h]
		\centering
		\includegraphics[scale=#1]{#2}
	\end{figure}
}     
            
    
    \newcommand{\twierdzonko}[1]{
        \begin{center}
        \begin{mdframed}
        #1
        \end{mdframed}          
        \end{center}
    } 
    
    \newcommand{\dwanajeden}[2]{
	\ensuremath \left( \begin{array}{c}
		#1\\
		#2
	\end{array} \right)
}  
      
%Stopka i head (sekcja której nie powinno się zmieniać)
    \pagestyle{fancy}
    \fancyhead{}
    \fancyfoot{}
    
    %Zmieniaj od tego miejsca
	\rfoot{\thepage}
	\lfoot{Grzegorz Koperwas, \LaTeX note}
	\renewcommand{\headrulewidth}{0pt}
	\renewcommand{\footrulewidth}{1pt}

    
\begin{document}

\section{27.10.2020}

\subsection{zadanie 1.}
\begin{enumerate}
	\item Prawda
	\item Prawda dla $e = 1$
	\item Prawda dla $e = 1$
	\item Fałsz
	\item \begin{gather*}
		      x + y = 5 \\
		      x^2 + y^2 = (x + y)^2 - 2xy = 25 - 2xy\\
		      25 - 2xy = 1\\
		      2xy = 24 \\
		      xy = 12, x \not = 0 \\
		      y = \frac{12}{x} \wedge x + y = 5 \rightarrow y = 5 -x\\
		      5x - x^2 = 12\\
		      x^2 -5x + 12 =0 \\
		      \Delta = 25 - 4 \cdot 12 < 0 \Rightarrow \text{Zdanie jest fałszywe}
	      \end{gather*}
	\item Fałsz dla $x= -1$
\end{enumerate}

\subsection{Zadanie 2}
\begin{enumerate}
	\item $\bigwedge y, z \left(x = y \cdot z \Rightarrow y = x \vee y = 1\right)$
	\item $\bigvee y, z \in \mathbb{N} \left(x \cdot y = z \Rightarrow \neg \left(\bigwedge w \in \mathbb{N} \left(w \cdot z = y \wedge w > x\right)\right)\right)$
\end{enumerate}

\subsection{Zadanie 3/4}


\clearpage
\section{19.01.2021}

\begin{gather*}
	\left| z \right| \left(\cos \alpha + i \sin \alpha \right) = \left|z\right| e^{i\alpha} \\
	\left| z \right| = \sqrt{a^2 + b^2}\\
	\left\{\begin{matrix}
		\cos \alpha = \frac{a}{\left| z \right|} \\
		\sin \alpha = \frac{b}{\left| z \right|}
	\end{matrix}\right.
\end{gather*}

\[\left| z \right| \left(\cos \alpha + i \sin \alpha \right) \cdot \left| w \right| \left(\cos \beta + i \sin \beta \right) = \left| z w \right| \left(\cos \left(\alpha + \beta\right) + i \sin \left(\alpha + \beta\right)\right)\]
\[\frac{\left| z \right| \left(\cos \alpha + i \sin \alpha\right)}
{\left| w \right| \left(\cos \beta + i \sin \beta \right)} = \left|\frac{z}{w}\right| \left(\cos \left(\alpha - \beta\right) + i \sin \left(\alpha - \beta\right)\right)\]

\[\left(\left| z \right| \left(\cos \alpha + i \sin \alpha \right) \right)^n = \left| z^n \right| \left(\cos \left(\alpha \cdot n\right) + i \sin \left(\alpha \cdot n\right)\right)\]
\end{document}
