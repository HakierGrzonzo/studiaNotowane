\documentclass[12pt,a4paper]{article}
\usepackage[polish]{babel}
\usepackage[T1]{fontenc}
\usepackage[utf8x]{inputenc}
\usepackage{hyperref}
\usepackage{url}
\usepackage[]{algorithm2e}
\usepackage{listings}

\usepackage{color}
\usepackage{listings}

\lstloadlanguages{% Check Dokumentation for further languages ...
	C,
	C++,
	csh,
	Java
}

\definecolor{red}{rgb}{0.6,0,0} % for strings
\definecolor{blue}{rgb}{0,0,0.6}
\definecolor{green}{rgb}{0,0.8,0}
\definecolor{cyan}{rgb}{0.0,0.6,0.6}

\lstset{
	language=csh,
	basicstyle=\footnotesize\ttfamily,
	numbers=left,
	numberstyle=\tiny,
	numbersep=5pt,
	tabsize=2,
	extendedchars=true,
	breaklines=true,
	frame=b,
	stringstyle=\color{blue}\ttfamily,
	showspaces=false,
	showtabs=false,
	xleftmargin=17pt,
	framexleftmargin=17pt,
	framexrightmargin=5pt,
	framexbottommargin=4pt,
	commentstyle=\color{green},
	morecomment=[l]{//}, %use comment-line-style!
	morecomment=[s]{/*}{*/}, %for multiline comments
	showstringspaces=false,
	morekeywords={ abstract, event, new, struct,
		as, explicit, null, switch,
		base, extern, object, this,
		bool, false, operator, throw,
		break, finally, out, true,
		byte, fixed, override, try,
		case, float, params, typeof,
		catch, for, private, uint,
		char, foreach, protected, ulong,
		checked, goto, public, unchecked,
		class, if, readonly, unsafe,
		const, implicit, ref, ushort,
		continue, in, return, using,
		decimal, int, sbyte, virtual,
		default, interface, sealed, volatile,
		delegate, internal, short, void,
		do, is, sizeof, while,
		double, lock, stackalloc,
		else, long, static,
		enum, namespace, string},
	keywordstyle=\color{cyan},
	identifierstyle=\color{red},
}
\usepackage{caption}
\DeclareCaptionFont{white}{\color{white}}
\DeclareCaptionFormat{listing}{\colorbox{blue}{\parbox{\textwidth}{\hspace{15pt}#1#2#3}}}
\captionsetup[lstlisting]{format=listing,labelfont=white,textfont=white, singlelinecheck=false, margin=0pt, font={bf,footnotesize}}


\addtolength{\hoffset}{-1.5cm}
\addtolength{\marginparwidth}{-1.5cm}
\addtolength{\textwidth}{3cm}
\addtolength{\voffset}{-1cm}
\addtolength{\textheight}{2.5cm}
\setlength{\topmargin}{0cm}
\setlength{\headheight}{0cm}

\begin{document}
	
	\title{Języki Skryptowe\\\small{dokumentacja projektu ,,Hotele''}}
	\author{Grzegorz Koperwas}
	\date{\today}

	\maketitle
	\newpage
	\section*{Część I}
	\subsection*{Opis programu}

    W Bajtocji jest n miast połączonych zaledwie n − 1 drogami. Każda z dróg łączy bezpośrednio dwa miasta.
    Wszystkie drogi mają taką samą długość i są dwukierunkowe. Wiadomo, że z każdego miasta da się dojechać
    do każdego innego dokładnie jedną trasą, złożoną z jednej lub większej liczby dróg. Inaczej mówiąc, sieć dróg
    tworzy drzewo.

    Król Bajtocji, Bajtazar, chce wybudować trzy luksusowe hotele, które będą gościć turystów z całego świata.

    Król chciałby, aby hotele znajdowały się w różnych miastach i były położone w tych samych odległościach od
    siebie.

    Pomóż królowi i napisz program, który obliczy, na ile sposobów można wybudować takie trzy hotele w Baj-
    tocji.

	\subsection*{Instrukcja obsługi}

    Należy wykonać plik \texttt{./run.sh}, wygeneruje on automatycznie zestawy
    danych testowych, wykona plik \texttt{./projekt.py} dla nich, oraz
    wygeneruje skryptem \texttt{./raport.py} plik \texttt{./raport.html}.

	\subsection*{Dodatkowe informacje}

    Wymagania:

    \begin{enumerate}
            \item Biblioteka Jinja2
            \item Python3 (sprawdzanie działanie na wersji 3.9.7, wcześniejsze mogą nie
                działać)
    \end{enumerate}

	\newpage
	\section*{Część II}
	\subsection*{Opis działania} 
	
    Dane jest drzewo $n$ węzłów.

    W pierwszej fazie, dla każdego węzła jest twożony jest słownik zwracający
    dystans do dannego węzła (miasta), dla dannego węzła\footnote{W dalszej
    części pracy będe stosował te wyrażenia wymiennie}.

    W drugiej fazie, dla każdego miasta, twożony jest słownik zwracający listę
    wszystkich miast, których odległość jest równa, dla danej odległości.

    Następnie, dla każdej odległości\footnote{kluczy słownika}, jest zliczana
    liczba miast spełniających warunek zadania\footnote{W wynniku optymalizacji
    powtóżenia są eliminowane}.
	
	\subsection*{Algorytm}
    
    \subsubsection*{Generacja słownika Miasto $\rightarrow$ odległość}
    
    Niech dana jest funkcja $s$

	\begin{algorithm}[H]
		\caption{Algorytm drukowania informacji o liczbie parzystej/nieprarzystej.}
	\end{algorithm}

	\subsection*{Implementacja}
	Opis, zasada i działanie programu ze względu na podział na pliki, nastepnie	funkcje programu wraz ze szczegółowym opisem działania (np.: formie pseudokodu, czy odniesienia do równania)
	\begin{lstlisting}
	Tutaj wklejamy fragment kodu, ktory chcemy opisac 
	(bez polskich znakow).
	\end{lstlisting}
	\subsection*{Testy}
	Tutaj powinna pojawić się analiza uzyskanych wyników oraz wykresy/pomiary.
	
	\subsection*{Eksperymenty}
	Sekcję używamy gdy porównywaliśmy dwa lub więcej algorytmów, albo wykonywaliśmy jakies pomiery.
	
	Warto dodać jakies wykresy jako obraz, albo tabele z wynikami. 
	
	Wszyskie wyniki powinny być opisane/poddane komentarzowi i poddane analizie statystycznej.
	\newpage
	\section*{Pełen kod aplikacji}
\begin{lstlisting}
Tutaj wklejamy pelen kod. 
\end{lstlisting}
\end{document}
