% !TEX program = xelatex
%Wzór dokumentu
%tu zmień marginesy i rozmiar czcionki
\documentclass[a5paper,10pt]{article}
\usepackage{inputenc}[utf8]
\usepackage[margin=1.2cm]{geometry}
\usepackage[polish]{babel}

%Lepiej tego nie zmieniaj, jak co to dodawaj pakiety
\usepackage{titlesec}
\usepackage{titling}
\usepackage{fancyhdr}
\usepackage{mdframed}
\usepackage{graphicx}
\usepackage{amsmath}
\usepackage{amsfonts}
\usepackage{multicol}
\usepackage{multirow}
\usepackage{listings}
\usepackage{caption}
\usepackage{float}
\usepackage{pdfpages}
\usepackage{tikz}
	\usetikzlibrary{arrows}
	\usetikzlibrary{patterns}
	\usetikzlibrary{decorations.pathmorphing}
\usepackage{pgf}
\usepackage[section]{placeins}

\usepackage{pict2e}

\makeatletter
\newcommand{\odpowiada}{\mathrel{\mathpalette\raising@circles@eq\relax}}
\newcommand{\raising@circles@eq}[2]{%
  \vphantom{#1+}%
  \vbox{
    \settowidth\unitlength{$#1\mspace{2mu}$}%
    \offinterlineskip\m@th
    \ialign{##\cr
      \hfil\small@circle{#1}$#1\mspace{1.5mu}$\cr\noalign{\vskip0.5\unitlength}
      $#1=$\cr\noalign{\post@vskip{+}{#1}}
      $#1\mspace{1.5mu}$\small@circle{#1}\hfill\cr\noalign{\post@vskip{-}{#1}}
    }%
  }%
}
\newcommand{\small@circle}[1]{%
  \smash{%
    \begin{picture}(1,1)
    \small@linethickness{#1}
    \put(0.5,0.5){\circle{1}}
    \end{picture}%
  }%
}
\newcommand{\small@linethickness}[1]{%
  \linethickness{%
      \ifx#1\displaystyle 0.8\fontdimen8\textfont3\else
      \ifx#1\textstyle 0.8\fontdimen8\textfont3\else
      \ifx#1\scriptstyle0.8\fontdimen8\scriptfont3\else
      1\fontdimen8\scriptscriptfont3\fi\fi\fi
  }%
}
\newcommand{\post@vskip}[2]{%
  \expandafter\vskip\expanded{%
    #1\ifx#2\scriptscriptstyle0.9\else\ifx#2\scriptstyle0.6\else0.3\fi\fi\unitlength
  }%
}
\makeatother


%inny wygląd
%\usepackage{tgbonum}


\usepackage{hyperref}
\hypersetup{
    colorlinks=true,
    linkcolor=blue,
    filecolor=magenta,      
    urlcolor=cyan,
}

\urlstyle{same}
%Zmienne, zmień je!
\graphicspath{ {./ilustracje/} }
\title{Tworzenie i uruchamianie prostych kontenerów w środowisku  docker-compose}
\author{Grzegorz Koperwas}
\date{\today}

%lokalizacja polska (odkomentuj jak piszesz po polsku)

\usepackage{polski}
\usepackage[polish]{babel} 
\usepackage{indentfirst}
\usepackage{icomma} 

\brokenpenalty=1000
\clubpenalty=1000
\widowpenalty=1000    

%nie odkometowuj wszystkiego, użyj mózgu
%\renewcommand\thechapter{\arabic{chapter}.}
\renewcommand\thesection{\arabic{section}.}
\renewcommand\thesubsection{\arabic{section}.\arabic{subsection}.}
\renewcommand\thesubsubsection{\arabic{subsubsection}.}

%Makra

\newcommand{\obrazek}[2]{
\begin{figure}[h]
    \centering
    \includegraphics[scale=#1]{#2}
\end{figure}
}     

\newcommand{\stopnie}{\ensuremath{^{\circ}}}

\newcommand{\twierdzonko}[1]{
    \begin{center}
    \begin{mdframed}
    #1
    \end{mdframed}          
    \end{center}
} 

\newcommand{\dwanajeden}[2]{
\ensuremath \left( \begin{array}{c}
    #1\\
    #2
\end{array} \right)
}  

%Stopka i head (sekcja której nie powinno się zmieniać)
\pagestyle{fancy}
\fancyhead{}
\fancyfoot{}

%Zmieniaj od tego miejsca
\rfoot{\thepage}
\lfoot{}
\lhead{}
\rhead{}
\renewcommand{\headrulewidth}{1pt}
\renewcommand{\footrulewidth}{1pt}



\begin{document}

\begin{multicols}{2}
    \section{Pochodne}
    \begin{align*}
        \sqrt{x} &\rightarrow \frac{1}{2 \sqrt{x}}\\
        \sin x &\rightarrow \cos x\\
        \cos x &\rightarrow - \sin x\\
        \tg x &\rightarrow \frac{1}{\cos^2 x}\\
        ctg\, x &\rightarrow \frac{ -1}{\sin^2 x}\\
        \ln x &\rightarrow \frac{1}{x}\\
        \frac{1}{x} &\rightarrow \frac{ - 1}{x^2}\\
        f \left(g \left(x\right)\right) &\rightarrow f'\left(g\left(x\right)\right) \cdot g'\left(x\right)
    \end{align*}

    \section{Macierze}
    \begin{align*}
        A &= \left[\begin{matrix}
            a   & b     \\
            c   & d     \\
        \end{matrix}\right] = 
        \left[ \begin{matrix}
            a_{1,1} & a_{1,2} \\
            a_{2,1} & a_{2,2}
        \end{matrix} \right]\\
        A^{-1} &= \frac{1}{ad - bc}\left[\begin{matrix}
            d   & -b    \\
            -c  & a     \\
        \end{matrix}\right]\\
        A \cdot B &= \left[\begin{matrix}
            a_{1,1} b_{1,1} + a_{1,2} b_{2,1} + \dots & \cdots \\
            a_{2,1} b_{1,1} + a_{2,2} b_{2,1} + \dots & \cdots \\
            \vdots & \ddots
        \end{matrix}\right]
    \end{align*}
    

    \section{Równania}

    \subsection{Falsi}

    \[
        x_n = x_{n-1} - \frac{f\left( x_{n-1} \right)}{f\left( x_0 \right) -
        \left( x_{n-1} \right)} \left( x_0 - x_{n-1} \right)
    \]

    \subsection{stycznych}

    \begin{align*}
        x_{n+1} &= x_n - \frac{f\left( x_n \right)}{f'\left(x_n\right)}
    \end{align*}

    \subsection{Iteracji prostej}

    \begin{align*}
        x &= f\left( x \right) \\
        x_n &= f\left( x_{n-1} \right) 
    \end{align*}

    \section{Układy równań}
    \begin{itemize}
        \item $D$ - macierz diagonalna, odwrotność to wszystkie elementy do
            $x^{-1}$
        \item $L + U$ - to co nie jest w diagonalnej
        \item $b$ - wynik
    \end{itemize}

    \subsection{Newton}

    \[
        X_n = X_{n-1} - \underbrace{j\left( X_n-1
        \right)^{-1}}_{\text{jakobian}} \cdot
        \underbrace{F\left(X_{n-1}\right)}_{\text{f(x) = 0}}
    \]

    \subsection{Thomas}

    $a, b, c$ to są te fikuśne wektorki, $d$ to rozwiązanie.

    \[
        \begin{matrix}
            \beta_1 = - \frac{c_1}{b_1}     & \gamma_1 = \frac{a_1}{b_1} \\
            \beta_i = - \frac{c_i}{\alpha_i \beta_{i-1} + b_i} & \gamma_i =
            \frac{\alpha_i - \alpha_i \gamma_{i - 1}}{\alpha_i \beta_{i - 1} +
            b_i}\\
        \end{matrix}
    \]
    \[
        x_n = \beta_n \underbrace{x_{n+1}}_{\text{lub 0}} + \gamma_n
    \]

    \subsection{Jacobi}

    \[x^{i+1} = -D^{-1} \left(L + U\right) x^i + D^{-1} b \]


    \subsection{Gauss Seidl}

    \[
        x_k^n = - \frac{1}{a_{kk}} \left(
            \sum^{k-1}_{j=1} a_{kj} x^n_j + \sum^n_{j = k + 1} a_{kj} x_j^n -
            b_k
        \right)
    \]

    \section{Aproksymacje}

    \subsection{Lagrange}

    Sumujesz dla każdego punktu $\left(x_i, y_i\right)$, każdy inny punkt:

    \begin{align*}
        r &= \underbrace{\frac{x - x_k}{x_i - x_k}}_{\text{każdy k punkt}} \cdot
        \dots \\
        w &+= y_i \cdot r
    \end{align*}

    \subsection{Średniokwadratowa}

    Bazy są w formie: 
    \[
        \begin{matrix}
            \phi_0  & \phi_1 & \phi_n \\
            1       & x      & x^{n}
        \end{matrix}
    \]
    \begin{align*}
        \bar{f}_k &= \sum^n_{j=0} w\left( x_j \right) \underbrace{f\left( x_j
            \right)}_{y_j} \phi_k\left( x_j \right) \\
            d_{ki} &= d_{ik} = \sum^n_{j=0} w\left( x_j \right) \phi_i\left( x_j
            \right) \phi_k\left( x_j \right) \\
            D \cdot b &= \bar{F} \quad \text{b to wynik}
    \end{align*}


    \section{różniczkowe}

    Wynik to ma być $y'(x)$, za $y\left(x\right)$ podstawiasz wartość $y$.
    $y_0$ to zawsze $y\left( x_0 \right)$ które ci dają. $h$ to długość
    przedziału zadanego, $n$ to liczba przedziałów.

    \subsection{euler}

    \begin{align*}
        y_n &= y_{n - 1} + h \cdot f\left( x, y \right) \\
        x_n &= x_{n - 1} + h \\
    \end{align*}

    \subsection{Heun}

    \begin{align*}
        x_{n + 1} &= x_{n} + h \\
        m_n &= f(x_n, y_n) \\
        k_n &= f\left( x_n, y_n + hm_n \right)\\
        y_{n+1} &= y_n + \frac{h}{2}\left( m_n + k_n \right)
    \end{align*}
    
    \subsection{Zmodyfikowany Euler}

    \begin{align*}
        m_n &= f(x_n, y_n) \\
        k_n &= f\left( x_n + \frac{h}{2}, y_n + \frac{h}{2}m_n \right)\\
        y_n &= y_n + hk_n \\
        x_{n + 1} &= x_{n} + h \\
    \end{align*}

    \subsection{Rungy-Kutty}

    \begin{align*}
        x_{n+1} &= x_n + h \\
        k_1 &= f(x_n, y_n) \\
        k_n &= f\left( x_n + \frac{h}{2}, y_n + \frac{hk_{n-1}}{2} \right)
            \text{dla 2, 3} \\
        k_4 &= f\left( x_{n+1}, y_n + hk_3 \right)\\
        y_{n+1} &= y_n + \frac{h}{6}\left( k_1 + 2k_2 + 2k_3 + k_4 \right)
    \end{align*}

    \section{całki}

    $f_i$ to wartość zadanej funkcji w przedziale, im większe $i$ tym większe
    $x$ do funkcji, 0 to początek przedziału, max to koniec.

    \subsection{Proste trapezy}

    \[
        I \approx \frac{h}{2}\left(f_0 + f_1 \right)
    \]

    \subsection{Proste simpsonowe parabole}
    \[
        I \approx \frac{h}{3}\left( f_0 + 4 f_1 + f_2 \right)
    \]
    
    \subsection{trzech ósmych}
    \[
        I \approx \frac{3h}{8}\left(f_0 + 3f_1 + 3f_2 + f_3\right)
    \]

    Złożone wersje to dzielenie zadanego przedziału na $n$ wersji prostych i
    dodawanie ich.
\end{multicols}
\end{document}
