\documentclass[aspectratio=169]{beamer}
\usepackage{polski}
\usepackage{minted}
\usepackage{tikz}
	\usetikzlibrary{arrows}
	\usetikzlibrary{patterns}
	\usetikzlibrary{decorations.pathmorphing}

\usemintedstyle{solarized-light}

\usetheme{Stx}

\title{GENERATORY ORAZ\\ ASYNCHRONICZNOŚĆ W PYTHONIE}

\subtitle{Czym to jest i z czym się to je?}

\author{Grzegorz Koperwas}

\institute{STX Next}

\date{\today}

\begin{document}


\titleframe

\section{Generatory}
\subsection{Po co one nam są?}
\begin{frame}[fragile]{Przykładowy problem}

  \begin{columns}
    \begin{column}{0.4 \textwidth}

      Załóżmy, że dostaliśmy zadanie, w którym musimy odczytać dużą ilość danych,
      przetworzyć ją w jakiś sposób i kazać użytkownikowi podjąć jakąś decyzję
      co do ich wartości. \\~\\

      \only<2>{
        Jednak po oddaniu programu, nasz użytkownik zwrócił nam uwagę że program
        ładuje się bardzo długo...
      }


    \end{column}
    \begin{column}{0.5 \textwidth}
      
      \scriptsize
      \inputminted[linenos, stripall]{python}{./examples/naive.py}

    \end{column}

  \end{columns}

\end{frame}

\begin{frame}[fragile]{Jak możemy ten program przyspieszyć?}

      \inputminted[linenos, stripall]{python}{./examples/naive.py}

\end{frame}

\begin{frame}[fragile]{Pierwsze podejście}

  \begin{columns}
    \begin{column}{0.5 \textwidth}

      Spróbujmy pobierać dane po jednej wartości na raz.\\~\\

      \visible<2->{Aplikacja działa, ale pojawia się nam jakiś dziwny 
        \texttt{if} z warunkiem wyjścia z pętli
      }\\~\\

      \visible<3->{Czy da się ten kod napisać ładniej?}

    \end{column}
    \begin{column}{0.5 \textwidth}
      \scriptsize
      \inputminted[linenos]{python}{./examples/whilelooped.py}
    \end{column}
  \end{columns}
  
\end{frame}

\subsection{Czym jest generator}
\begin{frame}{Czym jest generator?}
  Z dokumentacji \texttt{python}a dowiemy się że:

  \pause

  \begin{block}{Dokumentacja}
    [Generator is] a function which returns a \textbf{generator iterator}. It 
    looks like a \textbf{normal function} except that it contains
    \mintinline{python}{yield}
    expressions for producing a series of values usable in a for-loop or that 
    can be retrieved one at a time with the \mintinline{python}{next()} function.
  \end{block}

  \pause

  Co to oznacza po polsku?
\end{frame}

\begin{frame}[fragile]{Czym funkcjo-podobnym czymś jest generator?}
  \begin{columns}
    \begin{column}{0.5\textwidth}
      \scriptsize
      \begin{minted}[linenos]{python}
def generator():
  # this will be a generator function
      \end{minted}
      \onslide<2->
      \begin{minted}[linenos,firstnumber=3]{python}
  yield "foo"
      \end{minted}
      \onslide<3->
      \begin{minted}[linenos,firstnumber=4]{python}
  yield "bar"
      \end{minted}
      \onslide<4->
      \begin{minted}[linenos,firstnumber=5]{python}


print(generator())
      \end{minted}

      \onslide<5->
      \texttt{<generator object generator at 0x7fa85e944200>}
    \end{column}
    \begin{column}{0.5\textwidth}
      \onslide<4->
      Spróbujmy uruchomić naszą ,,funkcje''\\~\\

      \onslide<5->
      Zamiast naszego tekstu dostaliśmy jakiś dziwny obiekt...
      
    \end{column}
  \end{columns}
\end{frame}

\begin{frame}[fragile]{Powrót do dokumentacji}
  \begin{columns}
    \begin{column}{0.5 \textwidth}
      Definicja mówiła coś o funkcji \mintinline{python}{next()}, spróbujmy jej
      użyć.\\~\\

      \onslide<2->
      Otrzymaliśmy naszą wartość \texttt{foo}, spróbujmy jeszcze raz użyć 
      \mintinline{python}{next()}.\\~\\

      \onslide<3->
      Pojawiło się nasze \texttt{bar}, co się stanie jeśli jeszcze raz użyjemy 
      \mintinline{python}{next()}?

    \end{column}
    \begin{column}{0.5\textwidth}
      \onslide<1->
      \scriptsize
      \inputminted[linenos]{python}{./examples/simpleGen.py}

      \onslide<2->
      \texttt{foo}

      \onslide<3->
      \begin{minted}[linenos, firstnumber=9]{python}
print(next(wierd_obj))
      \end{minted}
      \texttt{bar}

      \onslide<4->
      \begin{minted}[linenos, firstnumber=10]{python}
print(next(wierd_obj))
      \end{minted}
      \begin{minted}{text}
Traceback (most recent call last):
  File "simpleGen.py", line 10, in <module>
    print(next(wierd_obj))
StopIteration
      \end{minted}
    \end{column}
  \end{columns}
\end{frame}

\begin{frame}[fragile]{Generator jako iterator}
  \begin{columns}
    \begin{column}{0.5\textwidth}
      Nasza \emph{,,funkcja''} zwraca nam jakiś iterator, który zwraca po kolei 
      wszystkie wartości z wyrażeń \mintinline{python}{yield}, a po zakończeniu
      wyrzuca jakieś \mintinline{python}{Exception}. Dodajmy parę
      \texttt{print}ów by zobaczyć co się dzieje.
    \end{column}
    \begin{column}{0.3\textwidth}
      \scriptsize
      \begin{minted}[linenos]{python}
def generator():
  print("here")
  yield "foo"
  print("there")
  yield "bar"
  print("done")


gen = generator()
print(next(gen))
      \end{minted}
      
      \onslide<2->
      \begin{minted}{text}
here
foo
      \end{minted}
      
      \onslide<3->
      \begin{minted}[linenos, firstnumber=11]{python}
print(next(gen))
      \end{minted}
      \begin{minted}{text}
there
bar
      \end{minted}
      \onslide<4->
      \begin{minted}[linenos, firstnumber=12]{python}
print(next(gen))
      \end{minted}
      \begin{minted}{text}
done
Traceback (most recent call last):
[...]
      \end{minted}
    \end{column}
  \end{columns}
\end{frame}

\begin{frame}{To czym jest generator?}
  \begin{itemize}
    \item Nasza \emph{,,funkcja''} \texttt{generator} jest tak naprawdę
      konstruktorem.
    \pause
    \item Nasz kod jest wykonywany dopiero wtedy, kiedy poprosimy o nową
      wartość.
    \pause
    \item Czy możemy wykożystać naszą funkcje w pętli \texttt{for in}?
  \end{itemize}
\end{frame}

\subsection{Generatory w praktyce}
\begin{frame}[fragile]{Generator oraz pętle \texttt{for in}}
  \begin{columns}
    \begin{column}{0.5 \textwidth}
      Skorzystajmy zatem z naszej \emph{,,funkcji''} w pętli:
    \end{column}
    \begin{column}{0.5 \textwidth}
      \scriptsize
      \inputminted[linenos]{python}{./examples/geninfor.py}
      \pause
      \begin{minted}{text}
1
2
3
      \end{minted}
    \end{column}
  \end{columns}
\end{frame}

\begin{frame}[fragile]{A co jeśli stworzymy nieskończony generator?}
  \begin{columns}
    \begin{column}{0.5 \textwidth}
      \scriptsize
      \inputminted[linenos]{python}{./examples/silnia.py}
    \end{column}
    \begin{column}{0.5 \textwidth}
      \pause
      \begin{minted}{text}
1
1
2
6
24
120
720
5040
40320
362880
[...]
      \end{minted}
    \end{column}
  \end{columns}
\end{frame}

\subsection{Jak z generatora skorzystać w naszym programie}
\begin{frame}[fragile]{Jak możemy poprawić nasz kod z użyciem generatorów?}
  \begin{columns}
    \begin{column}{0.5 \textwidth}
      Możemy napisać generator zwracający przetworzone wartości, jedna po
      drugiej.
    \end{column}
    \begin{column}{0.5 \textwidth}
      \scriptsize
      \inputminted[linenos]{python}{./examples/generator.py}
    \end{column}
  \end{columns}
\end{frame}

\begin{frame}[fragile]{Jak możemy poprawić nasz kod z użyciem generatorów?}
  \begin{columns}
    \begin{column}{0.5 \textwidth}
      Możemy również przepisać naszą pętle która prosi użytkownika o podjęcie 
      decyzji na generator.
    \end{column}
    \begin{column}{0.5 \textwidth}
      \scriptsize
      \inputminted[linenos]{python}{./examples/moreGenerator.py}
    \end{column}
  \end{columns}
\end{frame}

\subsection{Podsumowanie}
\begin{frame}{Podsumowanie}
  Generatory pozwalają nam pisać kod który generuje kolejne wartości dopiero
  wtedy kiedy są potrzebne. Robi to przez przerywanie wykonywania programu i
  zwracanie wartości za pomocą słowa kluczowego \mintinline{python}{yield}.
\end{frame}

\section{Asynchroniczność}
\subsection{Nowy problem}
\begin{frame}{Nowy problem}{Program jest nadal wolny, ale inaczej...}
  Po oddaniu naszego programu przepisanego na nowo poznane generatory,
  doszły do nas słuchy, że program może szybciej się ładuje na początku, ale 
  użytkownicy muszą dłużej czekać na załadowanie się nowego przypadku.
  \pause

  Użycie superkomputera nie pomogło.
\end{frame}

\begin{frame}{Analiza problemu}
  \begin{figure}[h]
  \begin{tikzpicture}
    % Oś czasu
    \draw[->] (0, 0) -- (11, 0) node [below] {\scriptsize Czas};
    \foreach \i in {1, 2, 3, 4, 5, 6, 7, 8, 9, 10}
    {
      \draw[thin] (\i, -0.1) -- (\i, 0);
    }
    % legenda 
    \node[left] at (0, .75) {\scriptsize Program:};
    \node[left] at (0, .25) {\scriptsize Procesor:};
    % ładowanie
    \fill [pattern = north east lines, pattern color=stxgreen, draw=none, fill] (1, 0) rectangle (1.1, .5);
    \fill [pattern = north east lines, pattern color=stxgreen, draw=none, fill] (2.9, 0) rectangle (3, .5);
    \draw (1, .5) rectangle (3, 1);
    \node at (2, .75) {\scriptsize\texttt{getOneData()}};
    % przetwarzanie
    \fill [pattern = north east lines, pattern color=stxgreen, draw=none, fill] (3, 0) rectangle (5, .5);
    \draw (3, .5) rectangle (5, 1);
    \node at (4, .75) {\scriptsize\texttt{processData()}};
    % user
    \fill [pattern = north east lines, pattern color=stxgreen, draw=none, fill] (5, 0) rectangle (5.1, .5);
    \fill [pattern = north east lines, pattern color=stxgreen, draw=none, fill] (9.9, 0) rectangle (10, .5);
    \draw (5, .5) rectangle (10, 1);
    \draw[->] (10, .75) -- (10.5, .75) -- (10.5, 1.5) -- (.5, 1.5) -- (.5, .75) -- (1, .75);
    \draw[fill, color= white] (8.05, 0) rectangle (8.95, 1.6);
    \node[right] at (5.1, .75) {\scriptsize\texttt{askUserForAction()}};
    \foreach \i in {0, .5, 1, 1.5}
    {
      \draw[dashed] (8.10, \i) -- (8.95, \i);
    }
  \end{tikzpicture}
  \centering
\end{figure}

  \pause 
  Wnioski: 
  \begin{itemize}
    \item Nasz program spędza dużą część swojego czasu na czekaniu.
      \pause
    \item Cały okres oczekiwania, który kiedyś był na początku programu, został
      rozbity na mniejsze między akcjami dla użytkownika.
      \pause
    \item Czy możemy pobierać, przetwarzać, i pytać użytkownika o decyzje w tym 
      samym czasie?
  \end{itemize}
\end{frame}

\subsection{input-output}
\begin{frame}{Na co nasz program czeka?}
  Za każdym razem, kiedy nasz program chce czytać dane nie znajdujące się w
  jego pamięci, to musi prosić system operacyjny o ich odczytanie.
  W tym czasie wykonywanie naszego programu jest zablokowane.\\~\\
  \pause

  Nazywamy taki dostęp do danych \emph{synchronicznym}. \\~\\

  \pause
  Czy możemy uzyskiwać dostęp do naszych danych inaczej?
\end{frame}

\subsection{asyncio}
\begin{frame}{Asynchroniczne I/O}
  Gdy zapytamy wikipedię, czym jest asynchroniczne I/O, to otrzymamy taką
  definicję:\\~\\

  \begin{block}{Definicja}
    \textbf{Asynchronous I/O} is a form of input/output processing that permits 
    other processing to continue before the transmission has finished.
  \end{block}
  \pause

  Czy możemy zatem użyć asynchronicznego I/O w naszym programie?
\end{frame}

\begin{frame}[fragile]{Asynchroniczne I/O w \texttt{python}ie}
  \begin{columns}
    \begin{column}{.48\textwidth}
      Spróbujmy dopisać słówko \mintinline{python}{async} do funkcji i ją
      uruchomić.\\~\\

      \onslide<2->
      Tym razem dostaliśmy dwa ostrzeżenia oraz dziwny obiekt. Co starają się
      nam one przekazać i czym naprawdę jest nasza \emph{,,funkcja''}?
    \end{column}
    \begin{column}{.48\textwidth}
      \scriptsize
      \onslide<1->
      \inputminted[linenos]{python}{./examples/async.py}

      \onslide<2->
      \begin{minted}{text}
async.py:4: RuntimeWarning: coroutine 'main'
was never awaited
  print(main())
RuntimeWarning: Enable tracemalloc to get 
the object allocation traceback
<coroutine object main at 0x7f3c9d240200>
      \end{minted}
    \end{column}
  \end{columns}
\end{frame}

\begin{frame}{Asynchroniczne I/O w \texttt{python}ie}
  \begin{columns}
    \begin{column}{.48 \textwidth}
      Zobaczmy co robi przykład w dokumentacji modułu \texttt{asyncio}.\\~\\
      \onslide<3->
      Mamy tu:
      \begin{itemize}
        \item Jakiś specjalny \mintinline{python}{sleep()}
        \item Słowa kluczowe \mintinline{python}{async} oraz
          \mintinline{python}{await}
        \item Dziwne wywołanie funkcji za pomocą
          \mintinline{python}{asyncio.run()}
      \end{itemize}
    \end{column}
    
    \begin{column}{.48 \textwidth}
      \scriptsize
      \onslide<2->
      \inputminted[linenos]{python}{./examples/asyncHello.py}
    \end{column}
  \end{columns}
\end{frame}

\begin{frame}{Asynchroniczne I/O w \texttt{python}ie}
  \begin{columns}
    \begin{column}{.48 \textwidth}
      Zobaczmy kolejny przykład.\\~\\
      
      \onslide<2->
      Mamy to dwa razy \mintinline{python}{asyncio.sleep}, więc program powinien
      się wykonywać 5 sekund.\\~\\

    \end{column}
    
    \begin{column}{.48 \textwidth}
      \scriptsize
      \onslide<2->
      \inputminted[linenos]{python}{./examples/asyncTasks.py}
    \end{column}
  \end{columns}
\end{frame}

\begin{frame}{Coś się wykonuje na raz...}{Spróbujmy to zepsuć}
  \begin{columns}
    \begin{column}{.48 \textwidth}
      Zmodyfikujmy sobie nasz przykład.\\~\\
      
      \onslide<3->
      Słówko \mintinline{python}{await} robi jakieś rzeczy podobne 
      do \mintinline{python}{yield}. Wykonywanie programu się zatrzymuje w
      środku \emph{,,funkcji''}, jednak tym razem nie widać co i gdzie
      kontroluje nasz program.
    \end{column}
    
    \begin{column}{.48 \textwidth}
      \scriptsize
      \onslide<2->
      \inputminted[linenos]{python}{./examples/asyncTasks2.py}
    \end{column}
  \end{columns}
\end{frame}

\begin{frame}{Psucia ciąg dalszy}
  \begin{columns}
    \begin{column}{.48 \textwidth}
      Czym jest ta linijka z \mintinline{python}{await asyncio.gather()}?
      Spróbujmy się jej pozbyć...\\~\\

      \onslide<2->
      Co się tu dzieje?
    \end{column}
    
    \begin{column}{.48 \textwidth}
      \scriptsize
      \onslide<1->
      \inputminted[linenos]{python}{./examples/asyncTasks3.py}
    \end{column}
  \end{columns}
\end{frame}

\begin{frame}{Czym są te funkcje, że musimy je tak dziwnie uruchamiać?}
  \begin{columns}
    \begin{column}{.48 \textwidth}
      Tym razem do czynienia mamy z korutynami, czyli funkcjami których działanie
      może zostać przerwane. \\~\\

      \onslide<2->
      Wykonywanie jest przerywane kiedy używamy słówka
      \mintinline{python}{await}, lecz nie jest to taka pełna magia. \\~\\
      
      \onslide<3->
      Nie możemy poprostu napisać wszędzie w naszym programie
      \mintinline{python}{async}/\mintinline{python}{await}, \texttt{asyncio}
      musi mieć na co czekać...
  

    \end{column}
    
    \begin{column}{.48 \textwidth}
      \scriptsize
      \onslide<1->
      \inputminted[linenos]{python}{./examples/asyncTasks.py}
    \end{column}
  \end{columns}
\end{frame}

\subsection{Jak wykorzystać asyncio w naszym programie?}
\begin{frame}{Jak możemy z asyncio skorzystać?}
  Spójrzmy ponownie na analizę wykonywania naszego programu.
  \begin{figure}[h]
  \begin{tikzpicture}
    % Oś czasu
    \draw[->] (0, 0) -- (11, 0) node [below] {\scriptsize Czas};
    \foreach \i in {1, 2, 3, 4, 5, 6, 7, 8, 9, 10}
    {
      \draw[thin] (\i, -0.1) -- (\i, 0);
    }
    % legenda 
    \node[left] at (0, .75) {\scriptsize Program:};
    \node[left] at (0, .25) {\scriptsize Procesor:};
    % ładowanie
    \fill [pattern = north east lines, pattern color=stxgreen, draw=none, fill] (1, 0) rectangle (1.1, .5);
    \fill [pattern = north east lines, pattern color=stxgreen, draw=none, fill] (2.9, 0) rectangle (3, .5);
    \draw (1, .5) rectangle (3, 1);
    \node at (2, .75) {\scriptsize\texttt{getOneData()}};
    % przetwarzanie
    \fill [pattern = north east lines, pattern color=stxgreen, draw=none, fill] (3, 0) rectangle (5, .5);
    \draw (3, .5) rectangle (5, 1);
    \node at (4, .75) {\scriptsize\texttt{processData()}};
    % user
    \fill [pattern = north east lines, pattern color=stxgreen, draw=none, fill] (5, 0) rectangle (5.1, .5);
    \fill [pattern = north east lines, pattern color=stxgreen, draw=none, fill] (9.9, 0) rectangle (10, .5);
    \draw (5, .5) rectangle (10, 1);
    \draw[->] (10, .75) -- (10.5, .75) -- (10.5, 1.5) -- (.5, 1.5) -- (.5, .75) -- (1, .75);
    \draw[fill, color= white] (8.05, 0) rectangle (8.95, 1.6);
    \node[right] at (5.1, .75) {\scriptsize\texttt{askUserForAction()}};
    \foreach \i in {0, .5, 1, 1.5}
    {
      \draw[dashed] (8.10, \i) -- (8.95, \i);
    }
  \end{tikzpicture}
  \centering
\end{figure}


  Spróbujmy pobierać dane wtedy, kiedy czekamy na podjęcie decyzji przez
  użytkownika.
\end{frame}

\begin{frame}[fragile]{Pierwsze przepisanie}
  \scriptsize
  \inputminted[linenos]{python}{./examples/asyncProgram.py}
  \pause
  Czy możemy użyć tutaj paru generatorów?
\end{frame}

\begin{frame}[fragile]{Generatory oraz asyncio}
  \begin{columns}
    \begin{column}{.48 \textwidth}
      \texttt{Asyncio} pozwala nam pisać asynchroniczne generatory, lecz 
      nie możemy ich użyć bezpośrednio. \\~\\

      Możemy skorzystać z struktury kolejki zaimplementowanej przez
      \texttt{asyncio} by buforować dane między generatorem pobierającym dane 
      oraz generatorem pytającym się użytkownika o decyzję.
    \end{column}

    \begin{column}{.48 \textwidth}
      \tiny
      \vspace{-10mm}
      \inputminted[linenos]{python}{./examples/asyncQueue.py}
    \end{column}
  \end{columns}
\end{frame}

\subsection{Podsumowanie}
\begin{frame}{Podsumowanie}
  Asyncio pozwala nam na przerywanie wykonywania się programu podczas operacji 
  wejścia/wyjścia. \texttt{Event loop} zarządza wykonującym się kodem
  asynchronicznym i wykonuje wtedy inne fragmenty programu.
\end{frame}

\begin{frame}

  \vspace{3cm}

  \begin{center}
  Dziękuje za uwagę\\
  \end{center}
  
  \vspace{3cm}

  \begin{flushright}
    \scriptsize{
      \emph{Grzegorz \color{stxgreen}{Koperwas}}
    }
  \end{flushright}
\end{frame}

\lastframe

\end{document}
