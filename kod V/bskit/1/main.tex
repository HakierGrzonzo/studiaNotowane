\documentclass[12pt,a4paper]{article}

%\usepackage[T1]{fontenc}
\usepackage[polish]{babel}
\usepackage[utf8]{inputenc}
\usepackage{lmodern}
\selectlanguage{polish}
\usepackage{graphicx}
\usepackage{biblatex}
\usepackage{csquotes}
\addbibresource{bib.bib}

\begin{document}


\nocite{*}

\pagenumbering{gobble}
\clearpage
\begin{figure}[h]
\centering
\includegraphics{media/ps-logo.png}
\end{figure}
\hspace{3cm}
\begin{center}Sprawozdanie z modułu nr 1\end{center}
\begin{center}BSKiST 2022/2023\end{center}
\hspace{3cm}
\begin{center}\large\textbf{Bezpieczeństwo sieci komputerowych i systemów teleinformatycznych}\end{center}
\hspace{7cm}
\begin{flushright}Kierunek: Informatyka
\end{flushright}
\begin{flushright}Członkowie zespołu:
\par
\textit{Grzegorz Koperwas}
\end{flushright}
\vfill
\begin{center}Gliwice, 2022/2023\end{center}

\newpage
\pagenumbering{arabic}
\tableofcontents

\newpage
\section{Wprowadzenie}

\subsection{Zespół projektowy}

\begin{enumerate}
    \item Grzegorz Koperwas :: Wszystko
\end{enumerate}

\newpage

\subsection{Wprowadzenie}

W internecie mamy do czynienia z wieloma rodzajami zagrożeń dla naszej
infrastruktury teleinformatycznej, takimi jak:

\begin{itemize}
	\item Wirusy i malware
	\item Ataki hakerskie
	\item Włamania i nieautoryzowane dostępy
	\item Phishing i oszustwa internetowe
	\item Awarie i błędy systemowe
\end{itemize}

Jednak w celu ochrony przed wyżej wymienionymi zagrożeniami, możemy skorzystać 
z wielu metod, oraz możemy oceniać stan infrastruktury ochronnej na wiele
sposobów.

\section{Rozwinięcie}

\subsection{Rodzaje ochrony przed zagrożeniami.}

Istnieje wiele różnych, dobrze ze sobą współpracujących metod zabezpieczania
naszej infrastruktury, lub wykrywania nieautoryzowanego dostępu, kiedy takowy
nastąpi. Niektórymi z nich są:

\begin{enumerate}
    \item \textbf{Ochrona oprogramowania oraz urządzeń przed wirusami i malware} -
        Zapewnienie wykrywania nieautoryzowanego oprogramowania gdy już się
        pojawi na komputerze jest jedną z podstawowych technik zabezpieczania
        systemów teleinformatycznych. Nawet jeżeli oprogramowanie nie jest nie
        bezpieczne dla naszego komputera, nadal konieczne jest jego wykrycie w
        celu zapobiegnięcia dalszego rozprzestrzeniania się takiego
        oprogramowania.

        Taką sytuacją może być wirus przeznaczony dla systemów \texttt{windows}, 
        który może być przesyłany dalej przez serwer plików \texttt{samba}
        uruchomiony na systemie \texttt{GNU/Linux}. Programem mogącym zapobiegać
        takim sytuacjom jest na przykład \texttt{clamav}.
    \item \textbf{Zabezpieczenie dostępu i identyfikacji użytkowników} -
        Odpowiednia konfiguracja oraz wykonanie systemu teleinformatycznego
        uniemożliwia nieautoryzowany dostęp do naszego systemu
        teleinformatycznego. Użycie autoryzacji opartej na hasłach jest
        powszechnie przyjętym standardem, jednak posiada ono wiele wad
        wynikających z problemów spowodowanych obecnością członka gatunku
        \emph{homo sapiens} pomiędzy klawiaturą (lub innym urządzeniem
        \texttt{HID}) a krzesłem bądź fotelem.
        
        Z powyższego powodu, istnieje wiele alternatywnych rozwiązać, które
        wzmacniają tradycyjny proces autoryzacji hasłem przez dodanie nie
        stałych danych które użytkownik musi podać, lub poprzez wymagania od
        użytkownika posiadania określonych przedmiotów. Przykładami takich
        rozwiązań są:
        \begin{itemize}
            \item Google authenticator.
            \item Autoryzacja przez kod przesyłany kanałem trzecim. Na przykład 
              przez wiadomość SMS lub e-mail.
        \end{itemize}

        Inne rozwiązania skupiają się na obowiązku posiadania przez użytkownika 
        jakiejś informacji lub klucza. Może to być odcisk palca (biometria) lub 
        klucz na karcie inteligentnej.
    \item \textbf{Szyfrowanie danych i transmisji} - Wszystkie informacje
      wrażliwe powinny być przesyłane oraz przechowywane w sposób w który
      podmioty nieautoryzowane nie mają do nich dostępu.
    \item \textbf{Regularne aktualizacje} - Regularne instalowanie aktualizacji
      bezpieczeństwa zabezpiecza nas przed znalezionymi lukami bezpieczeństwa.
      Dodatkowo nowoczesne rozwiązania w zakresie wytwarzania oprogramowania
      potrafią wyeliminować wiele błędów zanim zostaną dostarczone
      użytkownikowi.
    \item \textbf{Monitoring i raportowanie incydentów bezpieczeństwa} -
      Monitorowanie naszej infrastruktury pozwala nam reagować jeśli nasze
      rozwiązania skupione na prewencji nie są sobie poradzić z nowymi rodzajami
      zagrożeń. Do tego celu mogą służyć różnego rodzaju zapory sieciowe nowego
      rodzaju.
\end{enumerate}

\subsection{Normy i standardy związane z bezpieczeństwem teleinformatycznym:}

Istnieje wiele standardów umożliwiających łatwą i częstą ocenę odporności danego
systemu teleinformatycznego na ataki i nieautoryzowany dostęp. Dodatkowo
istnieją różne regulacje prawne które zmuszają podmioty do stosowania
określonego poziomu zabezpieczeń przy przechowywaniu określonych informacji.

Niektórymi z nich są:

\begin{enumerate}
  \item ISO 27001 - System Zarządzania Bezpieczeństwem Informacji
  \item NIST SP 800-53 - Standardy Bezpieczeństwa w Systemach Rządu Federalnego Stanów Zjednoczonych
  \item ISO 27032 - Zarządzanie Bezpieczeństwem Informacji w Środowisku Cyfrowym
  \item RODO - Rozporządzenie Parlamentu Europejskiego i Rady (UE) 2016/679
\end{enumerate}

\end{document}
