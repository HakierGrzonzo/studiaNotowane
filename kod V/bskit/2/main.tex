\documentclass[12pt,a4paper]{article}

\usepackage[polish]{babel}
\usepackage[utf8]{inputenc}
\usepackage{lmodern}
\selectlanguage{polish}
\usepackage{graphicx}
\usepackage{biblatex}
\usepackage{csquotes}
\addbibresource{bib.bib}

\begin{document}


\nocite{*}

\pagenumbering{gobble}
\clearpage
\begin{figure}[h]
\centering
\includegraphics{media/ps-logo.png}
\end{figure}
\hspace{3cm}
\begin{center}Sprawozdanie z modułu nr 2\end{center}
\begin{center}BSKiST 2022/2023\end{center}
\hspace{3cm}
\begin{center}\large\textbf{Bezpieczeństwo sieci komputerowych i systemów teleinformatycznych}\end{center}
\hspace{7cm}
\begin{flushright}Kierunek: Informatyka
\end{flushright}
\begin{flushright}Członkowie zespołu:
\par
\textit{Grzegorz Koperwas}
\end{flushright}
\vfill
\begin{center}Gliwice, 2022/2023\end{center}

\newpage
\pagenumbering{arabic}
\tableofcontents

\newpage
\section{Wprowadzenie}

\subsection{Zespół projektowy}
Grzegorz Koperwas :: Wszystko

\newpage

\subsection{Wprowadzenie}

Omawianym w tej pracy rozwiązaniem dostarczającym usługi kryptograficzne będzie
program \texttt{GnuPG}. Dodatkowo zostanie omówiony mechanizm uwierzytelniania
użytkowników \texttt{PAM} w kernelu Linux'a.

Oprócz tego zostanie omówione wdrażanie wirtualnej sieci prywatnej w oparciu o
program \texttt{OpenVPN} na systemie \texttt{GNU/Linux}.



\section{Rozwinięcie}

\subsection{GnuPG}

Program \texttt{GnuPG} jest dostępny w praktycznie każdej dystrybucji systemu
\texttt{Gnu/Linux}, na przykład w dystrybucji \texttt{Arch} możemy zainstalować
go za pomocy komendy \texttt{pacman -S gpg}.

W celu korzystania z programu należy wygenerować lub dodać istniejący klucz
prywatny. By wygenerować klucz należy użyć opcji \texttt{--full-generate-key}.
Następnie należy wybrać algorytm do którego będzie służył klucz, jego wielkość
oraz jak długo powinien obowiązywać.

\texttt{GnuPG} pozwala nam dokonywać wszystkich powszechnych operacji
kryptograficznych, takich jak:
\begin{itemize}
  \item Symetryczne szyfrowanie plików za pomocą hasła.
  \item Szyfrowanie oraz podpisywanie asymetryczne, gdzie znając klucz publiczny
    odbiorcy możemy wysłać mu wiadomość którą tylko on może otworzyć.
  \item Zarządzanie prywatnymi oraz publicznymi kluczami - Możemy zarządzać oraz
    używać kluczy prywatnych zapisanych na kartach inteligentnych
    \texttt{OpenPGP} oraz zarządzać poziomem zaufania do zapisanych kluczy
    publicznych innych użytkowników. 

    \texttt{GnuPG} również implementuje rozwiązanie \emph{Web-of-trust}, które
    jest realizowane poprzez gromadzenie podpisów kluczy publicznych danej osoby
    przez innych użytkowników. Na przykład jeżeli ufamy osobie A, a osoba A ufa
    osobie B, to nasz system automatycznie ufa osobie B.
\end{itemize}

\subsection{PAM}

Kernel Linux'a zawiera wbudowany system zarządzania procesem autoryzacji
użytkowników. \texttt{Linux Pluggable Authentication Modules} pozwala na
zarządzanie jakie metody uwierzytelniania są wystarczające dla danych akcji.

\texttt{PAM} jest częścią kernela, więc instalacja nie jest konieczna.

Konfigurowanie \texttt{PAM} jest realizowane poprzez pliki umieszczone w
katalogu \texttt{/etc/pam.d/}. Dla każdego programu, który wymaga autoryzacji
użytkowników (np. Blokada ekranu \texttt{swaylock} czy polecenie \texttt{sudo})
jest dostępny plik, który pozwala na konfigurację metod uwierzytelniania dla
danego programu.

Przykładowo dla programu \texttt{swaylock} możemy wymagać autoryzacji poprzez
czytnik linii papilarnych, a dla deamona \texttt{sshd}, z powodu trudności z
obsługą biometrii zdalnie, możemy wymagać hasła oraz kodu z aplikacji
\emph{google authenticator}.

\subsection{OpenVPN}


OpenVPN jest oprogramowaniem VPN, które pozwala na bezpieczne połączenie z
siecią prywatną poprzez publiczne połączenie internetowe. Można go skonfigurować
w celu uzyskania dostępu do zasobów sieciowych i aplikacji, które normalnie są
niedostępne poza siecią. OpenVPN jest elastyczny i można go skonfigurować do
pracy z różnymi protokołami, takimi jak TCP i UDP, i jest w stanie pracować z
różnymi systemami operacyjnymi, w tym Windows, macOS, iOS i Android.

Możemy zainstalować zarówno serwer jak i klient OpenVPN komendą \texttt{pacman
-S openvpn}. Konfigurację serwera openvpn najlepiej jest utworzyć na podstawie
domyślnego, okomentowanego configu oraz umieścić ją w folderze
\texttt{/etc/openvpn/server}.

Odpowiednio skonfigurowany serwer uruchamiamy jako deamon za pomocą
\texttt{systemd enable --now openvpn-server@<config>}.

\end{document}
