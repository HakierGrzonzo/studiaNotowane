\documentclass[12pt,a4paper]{article}

%\usepackage[T1]{fontenc}
\usepackage[polish]{babel}
\usepackage[utf8]{inputenc}
\usepackage{lmodern}
\selectlanguage{polish}
\usepackage{graphicx}
\usepackage{biblatex}
\usepackage{csquotes}
\addbibresource{bib.bib}

\begin{document}


\nocite{*}

\pagenumbering{gobble}
\clearpage
\begin{figure}[h]
\centering
\includegraphics{media/ps-logo.png}
\end{figure}
\hspace{3cm}
\begin{center}Sprawozdanie z modułu nr 1\end{center}
\begin{center}BSKiST 2022/2023\end{center}
\hspace{3cm}
\begin{center}\large\textbf{Bezpieczeństwo sieci komputerowych i systemów teleinformatycznych}\end{center}
\hspace{7cm}
\begin{flushright}Kierunek: Informatyka
\end{flushright}
\begin{flushright}Członkowie zespołu:
\par
\textit{Grzegorz Koperwas}
\end{flushright}
\vfill
\begin{center}Gliwice, 2022/2023\end{center}

\newpage
\pagenumbering{arabic}
\tableofcontents

\newpage
\section{Wprowadzenie}

\subsection{Zespół projektowy}

\begin{enumerate}
    \item Grzegorz Koperwas :: Wszystko
\end{enumerate}

\newpage

\subsection{Wprowadzenie}

W internecie mamy do czynienia z wieloma rodzajami zagrożeń dla naszej
infrastruktury teleinformatycznej, takimi jak:

\begin{itemize}
	\item Wirusy i malware
	\item Ataki hakerskie
	\item Włamania i nieautoryzowane dostępy
	\item Phishing i oszustwa internetowe
	\item Awarie i błędy systemowe
\end{itemize}

Jednak w celu ochrony przed wyżej wymienionymi zagrożeniami, możemy skorzystać 
z wielu metod, oraz możemy oceniać stan infrastruktury ochronnej na wiele
sposobów.

\subsection{Rozwinięcie}

\subsection{Rodzaje ochrony przed zagrożeniami.}

Istnieje wiele różnych, dobrze ze sobą współpracujących metod zabezpieczania
naszej infrastruktury, lub wykrywania nieautoryzowanego dostępu, kiedy takowy
nastąpi. Niektórymi z nich są:

\begin{enumerate}
    \item \textbf{Ochrona oprogramowania oraz urządzeń przed wirusami i malware} -
        Zapewnienie wykrywania nieautoryzowanego oprogramowania gdy już się
        pojawi na komputerze jest jedną z podstawowych technik zabezpieczania
        systemów teleinformatycznych. Nawet jeżeli oprogramowanie nie jest nie
        bezpieczne dla naszego komputera, nadal konieczne jest jego wykrycie w
        celu zapobiegnięcia dalszego rozprzestrzeniania się takiego
        oprogramowania.

        Taką sytuacją może być wirus przeznaczony dla systemów \texttt{windows}, 
        który może być przesyłany dalej przez serwer plików \texttt{samba}
        uruchomiony na systemie \texttt{GNU/Linux}. Programem mogącym zapobiegać
        takim sytuacjom jest na przykład \texttt{clamav}.
    \item \textbf{Zabezpieczenie dostępu i identyfikacji użytkowników} -
        Odpowiednia konfiguracja oraz wykonanie systemu teleinformatycznego
        uniemożliwia nieautoryzowany dostęp do naszego systemu
        teleinformatycznego. Użycie autoryzacji opartej na hasłach jest
        powszechnie przyjętym standardem, jednak posiada ono wiele wad
        wynikających z problemów spowodowanych obecnością członka gatunku
        \emph{homo sapiens} pomiędzy klawiaturą (lub innym urządzeniem
        \texttt{HID}) a krzesłem bądź fotelem.
        
        Z powyższego powodu, istnieje wiele alternatywnych rozwiązać, które
        wzmacniają tradycyjny proces autoryzacji hasłem przez dodanie nie
        stałych danych które użytkownik musi podać, lub poprzez wymagania od
        użytkownika posiadania określonych przedmiotów. Przykładami takich
        rozwiązań są:
        \begin{itemize}
            \item Google authenticator
            \item Autoryzacja przez kod przesyłany kanałem trzecim
        \end{itemize}
	\item Szyfrowanie danych i transmisji
	\item Regularne aktualizacje
	\item Monitoring i raportowanie incydentów bezpieczeństwa.
\end{enumerate}


\section{Podsumowanie i wnioski}

\begin{itemize}
\item \textit{Podsumowanie}
\item \textit{Wnioski}
\end{itemize}

\newpage
\section{Spis literatury}

\printbibliography[heading=none] 

\end{document}
