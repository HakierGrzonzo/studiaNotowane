\documentclass[aspectratio=169]{beamer}

\usepackage{inputenc}[utf8]
\usepackage[polish]{babel}

%Lepiej tego nie zmieniaj, jak co to dodawaj pakiety
\usepackage{fancyhdr}
\usepackage{mdframed}
\usepackage{graphicx}
\usepackage{listings}
\usepackage{caption}
\usepackage{float}
\usepackage{hyperref}
\hypersetup{
    colorlinks=false,
    linkcolor=blue,
    filecolor=magenta,      
    urlcolor=cyan,
}
%\apptocmd{\frame}{}{\justifying}{} % Allow optional arguments after frame.

\urlstyle{same}
%Zmienne, zmień je!
\graphicspath{ {./ilustracje/} }
\title[BFC64]{Brainfuck Compiler 64 - Programowanie bardzo niskopoziomowe.}
\author{Grzegorz Koperwas}
\date{\today}

%lokalizacja polska (odkomentuj jak piszesz po polsku)

\usepackage{polski}
\usepackage[polish]{babel} 
\usepackage{indentfirst}
\usepackage{icomma} 
\usetheme{Warsaw}

\brokenpenalty=1000
\clubpenalty=1000
\widowpenalty=1000    

%nie odkometowuj wszystkiego, użyj mózgu
%\renewcommand\thechapter{\arabic{chapter}.}
\renewcommand\thesection{\arabic{section}.}
\renewcommand\thesubsection{\arabic{section}.\arabic{subsection}.}
\renewcommand\thesubsubsection{\arabic{subsubsection}.}

%Makra

\newcommand{\obrazek}[2]{
\begin{figure}[h]
    \centering
    \includegraphics[scale=#1]{#2}
\end{figure}
}     
        

\newcommand{\twierdzonko}[1]{
    \begin{center}
    \begin{mdframed}
    #1
    \end{mdframed}          
    \end{center}
} 

\newcommand{\dwanajeden}[2]{
\ensuremath \left( \begin{array}{c}
    #1\\
    #2
\end{array} \right)
}  

\captionsetup[figure]{name=Załącznik}
\begin{document}
\begin{frame}
    \titlepage
\end{frame}
\begin{frame}
    \tableofcontents
\end{frame}
\section{Jak napisać kompilator?}
\begin{frame}
    \frametitle{Jak napisać kompilator?}
    Zanim zaczniemy pisać kompilator musimy sobie odpowiedzieć na parę pytań

    \begin{description}
        \item[Czemu?] \pause By móc mówić że napisałem kompilator.
        \item[Czego?] \pause Brainfuck'a - języka stworzonego by twórca mógł napisać kompilator w 256 bajtach.
        \item[Na co?] \pause Commodore 64 - Ten 40 letni assembler nie będzie taki trudny.
    \end{description}
\end{frame}
\begin{frame}[fragile]
    \frametitle{Brainfuck 101}
    \begin{columns}
        \begin{column}{0.5\textwidth}
            \textbf{Brainfuck} - jak sama nazwa wskazuje nie jest zbytnio czytelnym językiem.
            
            Brainfuck nie posiada koncepcji zmiennej, zamiast tego oferuje nam \emph{dużą} taśmę z komórkami na zmienne liczbowe.

            \begin{lstlisting}[frame=L,basicstyle=\tiny\ttfamily,numbers=left]
++++++++
[
    >++++
    [
        >++>+++>+++>+<<<<-
    ]
    >+>+>->>+
    [<]
    <-
]
>>.>---.+++++++..+++.>>.<-.<.+++.------.--------.>>+.>++.
            \end{lstlisting}
            \scriptsize{Jak ktoś pomyślał że to ,,Hello World'' to gratulacje.}

        \end{column}
        \begin{column}{.5\textwidth}
            Jego cała składnia składa się nie z słów, jak w \texttt{c++}, tylko z paru znaków: 

            \begin{description}
                \item[{$<$}, {$>$}] Przesuń taśmę w lewo/prawo
                \item[+, -] Dodaj/Odejmnij 1 od komórki pamięci
                \item[., ,] Wypisz/wczytaj znak do komórki pamięci
                \item[{[, ]}] Pętla \texttt{while~(komórka~!=~0)}.
            \end{description}
        \end{column}
    \end{columns}
\end{frame}
\end{document}
