% !TEX program = xelatex
%Wzór dokumentu
%tu zmień marginesy i rozmiar czcionki
    \documentclass[a4paper,12pt]{article}
	\usepackage{inputenc}[utf8]
    \usepackage[margin=2.5cm]{geometry}
    
 %Lepiej tego nie zmieniaj, jak co to dodawaj pakiety
	\usepackage{titlesec}
	\usepackage{titling}
	\usepackage{fancyhdr}
	\usepackage{mdframed}
	\usepackage{graphicx}
	\usepackage{amsmath}
	\usepackage{amsfonts}
	\usepackage{multicol}
	
%inny wygląd
	%\usepackage{tgbonum}
	
	
	%Zmienne, zmień je!
	\graphicspath{ {./ilustracje/} }
    \title{title}
    \author{Grzegorz Koperwas}
    \date{[data]}
    
  %lokalizacja polska (odkomentuj jak piszesz po polsku)
  
    \usepackage{polski}
    \usepackage[polish]{babel} 
    \usepackage{indentfirst}
	\usepackage{icomma} 
	
    \brokenpenalty=1000
    \clubpenalty=1000
    \widowpenalty=1000    
 
 %nie odkometowuj wszystkiego, użyj mózgu
    %\renewcommand\thechapter{\arabic{chapter}.}
	\renewcommand\thesection{\arabic{section}.}
	\renewcommand\thesubsection{\arabic{section}.\arabic{subsection}.}
	\renewcommand\thesubsubsection{\arabic{section}.\arabic{subsection}.\arabic{subsubsection}.}

%Makra
    
\newcommand{\obrazek}[2]{
	\begin{figure}[h]
		\centering
		\includegraphics[scale=#1]{#2}
	\end{figure}
}     
            
    
    \newcommand{\twierdzonko}[1]{
        \begin{center}
        \begin{mdframed}
        #1
        \end{mdframed}          
        \end{center}
    } 
    
    \newcommand{\dwanajeden}[2]{
	\ensuremath \left( \begin{array}{c}
		#1\\
		#2
	\end{array} \right)
}  
      
%Stopka i head (sekcja której nie powinno się zmieniać)
    \pagestyle{fancy}
    \fancyhead{}
    \fancyfoot{}
    
    %Zmieniaj od tego miejsca
	\rfoot{\thepage}
	\lfoot{Grzegorz Koperwas}
	\lhead{Numer Ćwiczeń: \thesection}
	\rhead{Ostatnia edycja: \today}
	\renewcommand{\headrulewidth}{1pt}
	\renewcommand{\footrulewidth}{1pt}

    
\begin{document}

\section{Pierwsze ćwiczenia:}


\subsection{Dlaczego ekonomia należy do nauk społecznych?}

Ekonomia należy do nauk społecznych ponieważ bada dystrybucje dóbr i zasobów w
\emph{społeczeństwie}, co jest jednym z procesów zachodzących w społeczeństwie.

\subsection{Z jakim okresem historycznym wiąże się naukowy rozwój ekonomii? Dlaczego?}

Ekonomia wyodrębniła się jako nauka pod koniec 18 wieku, wraz z początkami rewolucji
przemysłowej oraz po publikacji książki \emph{“The Wealth of Nations” Adama Smitha},
która to przyjeła teze iż bogactwo danego kraju nie wynika z pełności skarbca monarchy,
lecz z przychodu narodowego.

\subsection{Jakie impulsy spowodowały powstanie teorii ekonomicznych?}

Impulsem dla powstania Teorii Ekonomicznych była potrzeba opisania nowych procesów w rodzących się pod koniec 18 wieku systemach kapitalistycznych, które zastępowały agralne systemy feudalne. Przed ich powstaniem uważało się iż ziemia była głównym źródłem potęgi ekonomicznej kraju, gdyż system feudalny skupiał się właśnie na produkcji żywności z tejże ziemi.

Kolejnym z impulsów było upowszechnienie się narzędzi matematycznych oraz wczesne katastrofy ekonomiczne takie jak upadek \emph{Kompanii Mórz Południowych}.

\subsection{Porównaj definicje ekonomii \emph{A. Marshalla} ze współczesną jej definicją. Co łączy te definicje, co zaś różni?}

Definicja \emph{A. Marshalla} skupia się na relacji między ludźmi a dystrybucją zasobów materialnych. Nowoczesna ekonomia jest definiowana jako nauka społeczna o produkcji, dystrybucji oraz konsumpcji dóbr. Definicje te łączy uznanie ekonomi jako nauki badającej społeczeństwo jak i jego interakcje z dobrami, jednak \emph{definicja Marshalla} skupia się bardziej na badaniach społecznych, gdzie nowoczesna definicja skupia się bardziej na badaniu obiegu dóbr.

\subsection{Które [z] poniższych zagadnień mają charakter mikroekonomiczny, które zaś makroekonomiczny:}

\begin{enumerate}
	\item ,,Dlaczego oniżyła się cena jabłek?''

	      mikroekonomiczny

	\item ,,Dlaczego wzrósł poziom cen w Polsce,''

	      makroekonomiczny

	\item ,,Dlaczego wzrosło bezrobocie w Zabrzu''

	      mikroekonomiczny

	\item ,,Dlaczego wzrosło bezrobocie w Polsce''

	      makroekonomiczny

	\item ,,Dlaczego zmniejszyły się zyski w przemyśle tytoniowym''

	      mikroekonomiczny
\end{enumerate}

\subsection{Podaj inne przykłady problemów mikroekonomicznych i makroekonomicznych? (po 5 przykładów)}

\subsubsection*{Makroekonomiczne:}
\begin{itemize}
	\item Dlaczego zarobkki Amerykańskiej klasy średniej spadają;
	\item Dlaczego wzrosła produktywność społeczeństwa;
	\item Jakie będą skutki załamania giełdy amerykańskiej;
	\item Obniżenie stóp procentowych;
	\item Spadek wzrostu PKB
\end{itemize}
\subsubsection*{Mikroekonomiczne:}

\begin{itemize}
	\item Spadek zarobków jednej rodziny.
	\item Załamanie się cen najmu przestrzeni sklepowej w Nowym Yorku.
	\item Drastyczne zwiększenie się cen nowych kart graficznych jednej firmy na wskutek ogromnego zapotrzebowania.
	\item Kolejne rewizje podręcznika wpływające negatywnie na rynek wtórny.
	\item Wygranie przetargu przez jedną firmę.
\end{itemize}

\subsection{Które z poniższych zdań są zdaniami ekonomii pozytywnej, które zaś ekonomii normatywnej:}

\begin{itemize}
	\item ,,ceny zabawek dla dzieci są za wysokie''

	      Ekonomia normatywna.

	\item ,,wzrost ceny jabłek powoduje zmniejszenie ilości nabywanych jabłek''

	      Ekonomia pozytywna.

	\item ,,Nakłady na badania naukowe są zbyt małe''

	      Ekonomia normatywna.

	\item ,,Wymiar podatku od dochodów osobistych w Polsce osłabia zainteresowanie staraniami o wzrost dochodów''?

	      Ekonomia pozytywna.
\end{itemize}

\subsection{Ustal, jaki charakter mają podane wypowiedzi - normatywny czy pozytywny:}

\begin{enumerate}
	\item ,,Popieram podniesienie płacy minimalnej, ponieważ pomoże to pracownikom niewykfalikowanym''

	      Normatywny, ponieważ autor wypowiedzi wyraża swoją subiektywną, ogólną opinię. Nie  uwzględnia on innych skutków \emph{podniesienia płacy minimalnej.}

	\item ,,Sprzeciwiam się podniesieniu płacy minimalnej, ponieważ spowoduje to wzrost bezrobocia w grupie młodych i niewykfalikowanych robotników''

	      Normatywny, ponieważ autor wyraża swoją subiektywną opinię, jednak popiera ją faktami z ekonomii pozytywnej.
\end{enumerate}

\subsection{Czy można uniknąć sądów oceniających zjawiska i procesy gospodarcze?}

Nie można uniknąć sądów nad zjawiskami i procesami ekonomicznymi, ponieważ ekonomia jako nauka społeczna, bada społeczeństwo. Jako iż te zjawiska są częścią procesów zachodzących w społeczeństwie, naturalne jest to iż społeczeństwo będzie miało swoje zdanie na ich temat, zatem będzie oceniać je według swoich subiektywnych opinii.

\subsection{Jak powinien być skonstułowany model domu studenckiego, jeżeli celem badawczym jest:}

\subsubsection*{Sposób powstawiania decyzji o wysokości miesięcznej opłaty za miejsce:}

Model \emph{sposobu powstawania decyzji o wysokości miesięcznej opłaty za miejsce} w domu studenckim składa się z:

\begin{itemize}
	\item Kosztów stałych utrzymania domu studenckiego.
	\item Regulacji prawnych regulujączych opłaty oraz ulgi.
	\item Zapotrzebowania na miejsca.
	\item Stan domu studenckiego, lub/i kwota potrzebna na jego poprawę.
\end{itemize}

\twierdzonko{\paragraph{Odpowiedź:}
	Model składa się z zarządu podejmującego decyzję o wysokości opłaty.

	\vspace{0.6cm}
}

\subsubsection*{Opisanie sposobu rozwiązywania konfliktów między mieszkańcami:}

Model \emph{sposobu rozwiązywania konfliktów między mieszkańcami} w domu studenckim składa się z mieszkańców próbujączych rozwiązywać swoje konflikty według regulaminu.

\subsection{Wróć do pytania 7 i odpowiedz, czy zdania, które uznałeś za zdania ekonomii pozytywnej, można bez zastrzeżeń uznać za twierdzenia naukowe?}

Zdania, które uznałam za zdania ekonomi pozytywnej można bez problemu uznać za twierdzenia naukowe, gdyż są tezami sugerującymi związek między zjawiskami które można mierzyć.

\subsection{Porównaj następujące zdania i wskaż, które jest poprawne pod względem naukowym:}


\begin{enumerate}
	\item ,,jeżeli rolnicy wydajnie pracują i jest urodzaj, to ogólny dochód rolników się zmniejszy''

	\item ,,jeżeli rolnicy wydajnie pracują i jest urodzaj, to ogólny dochód rolników prawdopodobnie się zmniejszy''
\end{enumerate}

Zdanie drugie jest poprawne pod względem naukowym gdyż stawia hipotezę. Zdanie pierwsze sugeruje istnienie powiązania nie dając na nie dowodów.

\subsection{Wyjaśnij paradoks zawarty w zdaniach z pytania 12 i ustal, która z zasadzek myślenia ekonomicznego została ominięta.}

W zdaniach z pytania 12 został zawarty paradoks w którym mimo iż rolnicy mają bardziej sprzyjające warunki, to ich dochody nie urosną. Jest to spowodowane tym że w czasach nieurodzaju plony rolne są towarami deficytowymi, bardziej rzadkimi, a zapotrzebowanie na nie pozostaje niezmienione.

Została ominięta zasadzka o tym iż zwiększona podaż wpłynie na popyt.

\subsection{Załóżmy, że celem działalności gospodarczej jest polepszenie jakości życia. Jak można mierzyć stopień realizacji tego celu?}

Mierzeniem jakości życia zajmuje się na przykład indeks ,,HDI\footnote{Human Development Index}'' w którym oprócz prognozowanej długości życia oraz dostępu do edukacji uwzględnia się dochody netto populacji oraz jej siłę nabywczą.

\subsection{,,Teoria musi opisywać rzeczywistość''. Zajmij stanowisko wobec tego zdania i uzasadnij je.}

Ekonomia, jako nauka społeczna, podlega pewnym rygorom metody naukowej. Z tego powodu poważne teorie powinny być oparte na obserwacjach, tutaj społeczeństwa. Z takiego rozumowania możemy łatwo dojść do konkluzji iż \emph{Teoria musi opisywać rzeczywistość}.

\subsection{Ustal, które z następujących zdań jest hipotezą:}

\begin{itemize}
	\item ,,koty widzą w ciemności lepiej niż psy''

	      Nie jest to hipoteza

	\item ,,ludzie, którzy się częściej opalają, są bardziej narażeni na chorobę nowotworową skóry''

	      Nie jest to hipoteza

	\item ,,jeżeli dochody rosną, to ludzie konsumują więcej dóbr''

	      Jest to hipoteza

	\item ,,jeżeli ludzie uwierzą, że dany produkt jest szkodliwy dla zdrowia, to będą go mniej konsumować''

	      Jest to hipoteza
\end{itemize}

\subsection{Czy przy formułowaniu hipotez musi być stosowana zasada ceteris paribus?}

Zasada \emph{Ceteris paribus} musi być stosowana przy formułowaniu hipotez po to by oddzielić zmienne które mogą wpływać na wynik, od zmiennych które dana hipoteza opisuje.

\subsection{Dlaczego modele ekonomiczne należy traktować poważnie, lecz nie dosłownie?}

Modele ekonomiczne są uproszczeniem procesów zachodzących w gospodarce, a nie jej dokładną symulacją. Z tego powodu powiliśmy brać wyniki analiz za pomocą ich przeprowadzonych pod uwagę, lecz musimy pamiętać iż nikt nie jest w stanie w pełni przewidzieć przyszłości.

\subsection{Czy politycy mogą ignorować teorie ekonomiczne?}

Politycy mogą ignorować teorie ekonomiczne jednak muszą pamiętać o konsekwencjach swoich wyborów. Tak jak ja mogę nie zrobić zadań z \emph{podstaw ekonomii}, tak polityk może zacząć ignorować teorie ekonomiczne, lub powoływać się na nieprawdziwe teorie\footnote{Na przykład \emph{trickle down economics}}, jednak musi on pamiętać że z jego wyborów rozliczą go wyborcy lub inne wpływowe osoby.

\subsection{,,Teorie raz uznane za prawdziwe są zawsze uznawane za prawdziwe''. Czy zgadzasz się z tym stwierdzeniem.}

Tak jak zmienienie swojej opinii na dany temat nie jest niczym złym, tak zmienienie opinii na temat danej teorii przez środowisko naukowe nie jest niczym nadzwyczajnym.

Wiele teorii było kiedyś uznawanych za prawdziwe, czy mówimy tu o \emph{płaskiej ziemi}, o tym czy \emph{rząd nie powinien się w jakimkolwiek stopniu angażować w gospodarkę} czy \emph{merkantyliźmie}, jednak obecnie uznaje się je za obalone.

\end{document}
