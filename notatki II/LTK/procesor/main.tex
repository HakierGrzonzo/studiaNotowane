\documentclass[aspectratio=169]{beamer}

\newif\ifmore
\morefalse

\usepackage{inputenc}[utf8]
\usepackage[polish]{babel}

%Lepiej tego nie zmieniaj, jak co to dodawaj pakiety
\usepackage{fancyhdr}
\usepackage{mdframed}
\usepackage{graphicx}
\usepackage{listings}
\usepackage{caption}
\usepackage{float}
\usepackage{tikz}
\usepackage{xcolor}
\usetikzlibrary{arrows}
\usepackage{hyperref}
\hypersetup{
    colorlinks=false,
    linkcolor=blue,
    filecolor=magenta,      
    urlcolor=cyan,
}
%\apptocmd{\frame}{}{\justifying}{} % Allow optional arguments after frame.

\urlstyle{same}
%Zmienne, zmień je!
\graphicspath{ {.} }
\title{Procesory}
\author{Grzegorz Koperwas \and Kamil Kowalczyk \and Kacper Nuckowski \and Weronika Chruszcz}
\date{\today}

%lokalizacja polska (odkomentuj jak piszesz po polsku)

\usepackage{polski}
\usepackage[polish]{babel} 
\usepackage{indentfirst}
\usepackage{icomma} 
\usetheme{Warsaw}

\brokenpenalty=1000
\clubpenalty=1000
\widowpenalty=1000    

%nie odkometowuj wszystkiego, użyj mózgu
%\renewcommand\thechapter{\arabic{chapter}.}
%\renewcommand\thesection{\arabic{section}.}
%\renewcommand\thesubsection{\arabic{section}.\arabic{subsection}.}
%\renewcommand\thesubsubsection{\arabic{subsubsection}.}

%Makra

\newcommand{\obrazek}[2]{
\begin{figure}[h]
    \centering
    \includegraphics[scale=#1]{#2}
\end{figure}
}     
        

\newcommand{\twierdzonko}[1]{
    \begin{center}
    \begin{mdframed}
    #1
    \end{mdframed}          
    \end{center}
} 

\newcommand{\dwanajeden}[2]{
\ensuremath \left( \begin{array}{c}
    #1\\
    #2
\end{array} \right)
}  

\lstdefinestyle{json}{
    frame=trBL,
    numbers=left,
    basicstyle=\ttfamily\tiny,
    keywordstyle=\color{blue},
    morekeywords={ Metadata }
}


\begin{document}
\begin{frame}
    \titlepage
\end{frame}

\section{Proces produkcji}
\subsection{Czym jest binning?}
\begin{frame}
    \begin{columns}
        \begin{column}{.5\textwidth}
            \obrazek{.1}{/home/hakiergrzonzo/Desktop/polibuda/notatki II/LTK/procesor/tsmc.jpg}
        \end{column}
        \begin{column}{.5\textwidth}
            \obrazek{.1}{/home/hakiergrzonzo/Desktop/polibuda/notatki II/LTK/procesor/binning.png}
        \end{column}
    \end{columns}
\end{frame}

\subsection{Co oznaczają nanometry?}
\begin{frame}
    \obrazek{.5}{/home/hakiergrzonzo/Desktop/polibuda/notatki II/LTK/procesor/process.png}
\end{frame}

\subsection{Nowe sposoby na stare problemy}
\begin{frame}
    \begin{columns}
        \begin{column}{.6\textwidth}
            \obrazek{.25}{/home/hakiergrzonzo/Desktop/polibuda/notatki II/LTK/procesor/ryzen.jpg}
        \end{column}
        \begin{column}{.4\textwidth}
            Procesor Ryzen 9 3900X ma 2 chipy z rdzeniami i jeden do obsługi i/o na starszym procesie.
        \end{column}
    \end{columns}
\end{frame}

\section{RISC a CISC}
\subsection{Czym jest CISC?}
\begin{frame}
    \begin{columns}
        \begin{column}{.3\textwidth}
            \obrazek{.09}{/home/hakiergrzonzo/Desktop/polibuda/notatki II/LTK/procesor/x86.png}
            \begin{center}
                \tiny{Lista kategorii instrukcji x86 w 2021}
            \end{center}
        \end{column}
        \begin{column}{.7\textwidth}
            \obrazek{.4}{/home/hakiergrzonzo/Desktop/polibuda/notatki II/LTK/procesor/x87.jpg}
            \begin{center}
                \tiny{Koprocesor 8087 do operacji na floatach}
            \end{center}
        \end{column}
    \end{columns}
\end{frame}

\subsection{Czym jest RISC?}
\begin{frame}
    \begin{columns}
        \begin{column}{.3\textwidth}
            \obrazek{.15}{/home/hakiergrzonzo/Desktop/polibuda/notatki II/LTK/procesor/riscV.jpg}
        \end{column}
        \begin{column}{.7\textwidth}
            \obrazek{.25}{/home/hakiergrzonzo/Desktop/polibuda/notatki II/LTK/procesor/arm.png}
        \end{column}
    \end{columns}
\end{frame}

\subsection{Jak mieć ciasto i zjeść ciasto naraz}
\begin{frame}
    \obrazek{.35}{/home/hakiergrzonzo/Desktop/polibuda/notatki II/LTK/procesor/microcode.png}
\end{frame}

\section{Herce, rdzenie, oraz pipelining}
\subsection{O czym decyduje zegar?}
\begin{frame}
    \begin{columns}
        \begin{column}{.7\textwidth}
            \obrazek{.4}{boost.png}
        \end{column}
        \begin{column}{.3\textwidth}
            Częstotliwość zależy od:
            \begin{itemize}
                \item Dostępnego chłodzenia i mocy
                \item Danego zadania i instrukcji
                \item Czasu (W procesorach intela)
            \end{itemize}
        \end{column}
    \end{columns}
\end{frame}

\subsection{Czym jest pipeline, i skąd biorą się te exploity?}
\begin{frame}
    \begin{columns}
        \begin{column}{.5\textwidth}
            \obrazek{.4}{pipeline.png}
        \end{column}
        \begin{column}{.5\textwidth}
            Dłuższy pipeline to wyższe zegary, ale długie pipeliny wymagają predykcji.
        \end{column}
    \end{columns}
\end{frame}

\subsection{Wincyj rdzenióf}
\begin{frame}
    \begin{columns}
        \begin{column}{.5\textwidth}
            \obrazek{.3}{cores.jpg}
        \end{column}
        \begin{column}{.5\textwidth}
            \begin{itemize}
                \item Więcej rdzeni - większe chipy
                \item Programowanie aplikacji wielowątkowych jest trudne
            \end{itemize}
        \end{column}
    \end{columns}
\end{frame}

\section{Obecne trendy}
\subsection{big.Little}
\begin{frame}
    \frametitle{Czym jest big.LITTLE}
    \begin{columns}
        \begin{column}{.3\textwidth}
            \obrazek{.23}{bigLittle.jpg}
            \begin{center}
                \tiny{4 duże rdzenie i 4 małe, zarządzane przez system.}
            \end{center}
        \end{column}
        \begin{column}{.7\textwidth}
            \obrazek{.12}{m1.jpg}
        \end{column}
    \end{columns}
\end{frame}

\subsection{Rdzenie, TDP i zegary}
\begin{frame}
    \begin{columns}
        \begin{column}{.6\textwidth}
            \obrazek{.3}{steamhw.png}
            \begin{center}
                \tiny{Z danych Steam'a}
            \end{center}
        \end{column}
        \begin{column}{.4\textwidth}
            \begin{itemize}
                \item Zegary ustabilizowane na około 4,5Ghz
                \item Agresywny boost (nawet do 250W)
                \item Koniec dominacji 4 rdzeni.
                \item Duże ilości pamięci Cache
            \end{itemize}
        \end{column}
    \end{columns}
\end{frame}

\begin{frame}
    \begin{center}
        \Large{Dziękujemy za uwagę!}
    \end{center}
\end{frame}

\end{document}
