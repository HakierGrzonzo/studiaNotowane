% !TEX program = xelatex
%Wzór dokumentu
%tu zmień marginesy i rozmiar czcionki
\documentclass[a4paper,12pt]{article}
\usepackage{inputenc}[utf8]
\usepackage[margin=1.5cm]{geometry}
\usepackage[polish]{babel}

%Lepiej tego nie zmieniaj, jak co to dodawaj pakiety
\usepackage{titlesec}
\usepackage{titling}
\usepackage{fancyhdr}
\usepackage{mdframed}
\usepackage{graphicx}
\usepackage{amsmath}
\usepackage{amsfonts}
\usepackage{multicol}
\usepackage{multirow}
\usepackage{listings}
\usepackage{caption}
\usepackage{float}
\usepackage{pdfpages}
\usepackage{tikz}
	\usetikzlibrary{arrows}
	\usetikzlibrary{patterns}
	\usetikzlibrary{decorations.pathmorphing}
\usepackage{pgf}
\usepackage[section]{placeins}

\usepackage{pict2e}

\makeatletter
\newcommand{\odpowiada}{\mathrel{\mathpalette\raising@circles@eq\relax}}
\newcommand{\raising@circles@eq}[2]{%
  \vphantom{#1+}%
  \vbox{
    \settowidth\unitlength{$#1\mspace{2mu}$}%
    \offinterlineskip\m@th
    \ialign{##\cr
      \hfil\small@circle{#1}$#1\mspace{1.5mu}$\cr\noalign{\vskip0.5\unitlength}
      $#1=$\cr\noalign{\post@vskip{+}{#1}}
      $#1\mspace{1.5mu}$\small@circle{#1}\hfill\cr\noalign{\post@vskip{-}{#1}}
    }%
  }%
}
\newcommand{\small@circle}[1]{%
  \smash{%
    \begin{picture}(1,1)
    \small@linethickness{#1}
    \put(0.5,0.5){\circle{1}}
    \end{picture}%
  }%
}
\newcommand{\small@linethickness}[1]{%
  \linethickness{%
      \ifx#1\displaystyle 0.8\fontdimen8\textfont3\else
      \ifx#1\textstyle 0.8\fontdimen8\textfont3\else
      \ifx#1\scriptstyle0.8\fontdimen8\scriptfont3\else
      1\fontdimen8\scriptscriptfont3\fi\fi\fi
  }%
}
\newcommand{\post@vskip}[2]{%
  \expandafter\vskip\expanded{%
    #1\ifx#2\scriptscriptstyle0.9\else\ifx#2\scriptstyle0.6\else0.3\fi\fi\unitlength
  }%
}
\makeatother


%inny wygląd
%\usepackage{tgbonum}


\usepackage{hyperref}
\hypersetup{
    colorlinks=true,
    linkcolor=blue,
    filecolor=magenta,      
    urlcolor=cyan,
}

\urlstyle{same}
%Zmienne, zmień je!
\graphicspath{ {./ilustracje/} }
\title{Tworzenie i uruchamianie prostych kontenerów w środowisku  docker-compose}
\author{Grzegorz Koperwas}
\date{\today}

%lokalizacja polska (odkomentuj jak piszesz po polsku)

\usepackage{polski}
\usepackage[polish]{babel} 
\usepackage{indentfirst}
\usepackage{icomma} 

\brokenpenalty=1000
\clubpenalty=1000
\widowpenalty=1000    

%nie odkometowuj wszystkiego, użyj mózgu
%\renewcommand\thechapter{\arabic{chapter}.}
\renewcommand\thesection{\arabic{section}.}
\renewcommand\thesubsection{\arabic{section}.\arabic{subsection}.}
\renewcommand\thesubsubsection{\arabic{subsubsection}.}

%Makra

\newcommand{\obrazek}[2]{
\begin{figure}[h]
    \centering
    \includegraphics[scale=#1]{#2}
\end{figure}
}     

\newcommand{\stopnie}{\ensuremath{^{\circ}}}

\newcommand{\twierdzonko}[1]{
    \begin{center}
    \begin{mdframed}
    #1
    \end{mdframed}          
    \end{center}
} 

\newcommand{\dwanajeden}[2]{
\ensuremath \left( \begin{array}{c}
    #1\\
    #2
\end{array} \right)
}  

%Stopka i head (sekcja której nie powinno się zmieniać)
\pagestyle{fancy}
\fancyhead{}
\fancyfoot{}

%Zmieniaj od tego miejsca
\rfoot{\thepage}
\lfoot{}
\lhead{}
\rhead{Ostatnia edycja: \today}
\renewcommand{\headrulewidth}{1pt}
\renewcommand{\footrulewidth}{1pt}



\begin{document}

\begin{multicols}{2}
    \section{Pochodne}
    \begin{align*}
        \sqrt{x} &\rightarrow \frac{1}{2 \sqrt{x}}\\
        \sin x &\rightarrow \cos x\\
        \cos x &\rightarrow - \sin x\\
        \tg x &\rightarrow \frac{1}{\cos^2 x}\\
        ctg\, x &\rightarrow \frac{ -1}{\sin^2 x}\\
        \ln x &\rightarrow \frac{1}{x}\\
        \frac{1}{x} &\rightarrow \frac{ - 1}{x^2}\\
        f \left(g \left(x\right)\right) &\rightarrow f'\left(g\left(x\right)\right) \cdot g'\left(x\right)
    \end{align*}
    \section{Całki}

    \begingroup
    \allowdisplaybreaks
    \begin{align*}
        \int x^a\, dx &= \frac{x^{a + 1}}{a + 1} + c \\
        \int \frac{dx}{x} &= \ln \left| x \right| + c \\
        \int \frac{dx}{x^2 + a_2} &= \frac{1}{a} \arc\tan \frac{x}{a} + c \\
        \int \frac{dx}{\sqrt{a^2 - x^2}} &= \arc\sin \frac{x}{a} + c \\
        \int e^x\, dx &= e^x + c \\
        \int a^x\, dx &= \frac{a^x}{\ln a} + c \\ 
        \int \cos x\, dx &= \sin x + c \\
        \int \sin x\, dx &= - \cos x + c \\
        \int \frac{dx}{\cos^{2} x} &= \tan x + c \\
        \int \frac{dx}{\sin^{2} x} &= - ctg\,  x + c \\
        \int \frac{dx}{ax + b} &= \frac{1}{x} \ln \left| ax + b \right| + c \\
        \int \tan x\, dx &= -\ln \left| \cos x \right| + c \\
        \int ctg\,  x\, dx &= \ln \left| \sin x \right| + c \\
        \int \left(ax + b\right)^n\, dx &= \frac{\left(ax + b\right)^{n + 1}}{a\left(n + 1\right)} + c \\
        \int \frac{dx}{a^2 - x^2} &= \frac{1}{2a} \ln \left| \frac{a + x}{a - x} \right| + c \\
        \int \sqrt{a^2 - x^2}\, dx &= \frac{a^2}{2} \arc\sin \frac{x}{a} + \frac{x}{2}\sqrt{a^2 - x^2} + c
    \end{align*}
    \endgroup
    \subsection*{Całkowanie przez części}
    \[
        \int f\left( x \right) \cdot g' \left( x \right)\, dx = f\left( x \right)\cdot g\left( x \right) - \int f' \left( x \right) \cdot g\left( x \right)\, dx
    \]

    \subsection*{Całkowanie przez podstawianie}
    \[
        \int \limits^b_a f\left(g \left(x\right)\right) \cdot g' \left(x\right)\, dx = \int \limits^{g\left(b\right)}_{g\left(a\right)} f \left(t\right)\, dt,\quad t = g \left(x\right)
    \]
    \section{Fourier}
    \[
        f\left(x\right) = \frac{a_0}{2} + \sum \limits^\infty_{n = 1} \left( a_n \cos \frac{2\pi nx}{T} + b_n \sin \frac{2\pi nx}{T} \right)
    \]
    Gdzie:
    \begin{align*}
        a_0 &= \frac{1}{T} \int \limits^T_T f \left( x \right)\, dx \\
        a_n &= \frac{1}{T} \int \limits^T_T f \left( x \right) \cos \frac{\pi n x}{T}\, dx \\
        b_n &= \frac{1}{T} \int \limits^T_T f \left( x \right) \sin \frac{\pi n x}{T}\, dx
    \end{align*}

    \subsection*{Tożsamość Parsevala}
    \[
        \frac{1}{\pi}\int \limits^\pi_{ -\pi} f^2 \left(x\right)\, dx = \frac{a_0^2}{2} + \sum \limits^\infty_{n = 1} \left( a^2_n + b^2_n \right)
    \]

    \section{funkcja dwóch zmiennych}
    Jeśli $f^"_{xy} \not = f^"_{yx}$ to zjebałeś.

    \[
        W\left(P\right) = \left|
        \begin{matrix}
            f^"_{x x}\left(P\right) & f^"_{x y}\left(P\right) \\
            f^"_{x y}\left(P\right) & f^"_{y y}\left(P\right)
        \end{matrix}\right|
    \]

    Jeśli $W\left(P\right) > 0$ to ekstremum.

    Jeśli $W\left(P\right) < 0$ to nie ekstremum, w innym przypadku chuj wie.

    Jeżeli $f^"_{x x}\left(P\right) > 0$ to maksimum.

    Jeżeli $f^"_{x x}\left(P\right) < 0$ to minimum.

\section{Laplace}
    \[
        F\left(s\right) = \int \limits^\infty_0 f\left( t \right) e^{ - s t} \, dt
    \]

    \subsection*{Własności}

    \begin{align*}
        a f\left(t\right) + b g\left(t\right) &\odpowiada a F\left(s\right) + b G\left(s\right) \\
        f \left(a t\right) &\odpowiada \frac{1}{a} F \left(\frac{s}{a}\right) \\
        f \left(t - t_0 \right) H\left(t - t_0 \right) &\odpowiada e^{ - t_0 s} F\left(s\right), \quad t_0 > 0 \\
        e^{s_0 t} f \left( t \right) &\odpowiada F\left( s - s_0 \right) \\
    \end{align*}

    \subsubsection*{Transformata pochodnej}

    \begin{align*}
        f' \left( t \right) &\odpowiada s F \left( s \right) - f\left( 0^+ \right) \\
        f'' \left( t \right) &\odpowiada s^2 F\left(s \right) - sf \left( 0^+\right) - f'\left(0^+\right)
    \end{align*}

    \subsubsection*{Pochodna transformaty}

    \[
        t f \left( t \right) \odpowiada - F'\left(s\right)
    \]

    \subsubsection*{Transformata całki}

    \[
        \int \limits^t_0 f\left( u \right) \, du \odpowiada \frac{F \left( s \right)}{s}
    \]

    \subsubsection*{Całka transformaty} 

    \[
        \frac{f \left( t \right)}{t} \odpowiada \int \limits^\infty_s F\left(u\right) \, du
    \]

    \subsection*{Transformaty podstawowych funkcji}

    \begingroup
    \allowdisplaybreaks
    \begin{align*}
        1 &\odpowiada \frac{1}{s} \\
        t &\odpowiada \frac{1}{s^2} \\
        t^n &\odpowiada \frac{n!}{s^{n + 1}} \\
% ok
        e^{s_0 t} &\odpowiada \frac{1}{s - s_0} \\
        t e^{s_0 t} &\odpowiada \frac{1}{\left( s - s_0\right)^2} \\
        t^n e^{s_0 t} &\odpowiada \frac{n!}{\left( s - s_0\right)^{n + 1}} \\
% ok
        \sin \omega t &\odpowiada \frac{\omega}{s^2 + \omega^2} \\
        \cos \omega t &\odpowiada \frac{s}{s^2 + \omega^2} \\
% ok
        t \sin \omega t &\odpowiada \frac{2 \omega s}{\left(s^2 + \omega^2\right)^2} \\
        t \cos \omega t &\odpowiada \frac{s^2 - \omega^2}{\left(s^2 + \omega^2\right)^2} \\
% ok
        e^{s_0 t} \sin \omega t &\odpowiada \frac{\omega}{\left( s - s_0 \right)^2 + \omega^2} \\
        e^{s_0 t} \cos \omega t &\odpowiada \frac{s - s_0}{\left( s - s_0 \right)^2 + \omega^2} \\
    \end{align*}
    \endgroup

    \href{https://www.math.purdue.edu/~caiz/MA527-cai/lectures/Table%20of%20Laplace%20Transforms.pdf}{Dodatkowe wzorki tu}

\section{Transformata odwrotna do Laplace'a}

    \[
        f\left(t\right) = \frac{1}{2 \pi i} \int \limits^{c + i\infty}_{c - i\infty} F \left( s \right) e^{s t} \, ds
    \]

\section{Sploty}

    \[
        f\left( t \right) * g\left( t \right) = \int \limits^t_0 f \left( u \right) g \left(t - u \right)\, du
    \]

    \subsection*{Twierdzenia}

    \begin{align*}
        f \left( t \right) * g \left( t \right) &\odpowiada F\left(s\right) \cdot G \left(s\right) \\
        \mathcal{L}^{ - 1} \left[ F \left( s \right) \cdot G \left( s \right) \right] &= f \left( t \right) * g \left( t \right) \\
        \left[ f \left(t\right) * g \left( t \right)\right]' &= f \left(t\right) * g' \left( t \right) + f \left( t \right) g \left( 0 \right) \\
        \left[ f \left(t\right) * g \left( t \right)\right]' &\odpowiada s F\left( s \right) G \left( s \right) \\
    \end{align*}

\section{transformata $\mathcal{Z}$}

\[
    F \left( s \right) = \sum \limits^\infty_{n = 0} f \left(n\right) e^{ - s n}\, dt
\]

Przyjmujemy $z = e^s$, zatem:

\[
    F \left( s \right) = \sum \limits^\infty_{n = 0} f \left(n\right) z^{ - n}\, dt
\]

\subsection*{Własności}

\begin{align*}
    a f\left( n \right) + b g\left( n \right) &\odpowiada a F\left( z \right) + b G\left( z \right) \\
    a^n f \left( n \right) &\odpowiada F\left(\frac{z}{a}\right) \\ 
    f \left( n - k \right) &\odpowiada z^{ - k} F \left( z \right), \quad n \leq k \\
    f \left(n + k \right) &\odpowiada z^k \cdot \left[F\left(z\right) - \sum \limits^{k - 1}_{i = 0} f\left(i\right) z^{ - i}\right]
\end{align*}

\subsubsection*{Transformata ciągu sum}

Jeżeli $g\left(n\right) = \sum \limits^n_{i = 0} f \left(i\right)$, to:

\[
    g \left(n\right) \odpowiada \frac{z}{z - 1} F\left(z\right)
\]

\subsubsection*{Różniczkowanie transformaty}

\[
    n f \left( n \right) \odpowiada - z F' \left( z \right)
\]

\subsection*{Twierdzenia}

\begin{align*}
    \lim_{z \rightarrow \infty} F \left( z \right) &= f \left( 0 \right) \\
    \lim_{z \rightarrow 1^+} \left(z - 1 \right) F \left( z \right) &= \lim_{n \rightarrow \infty} f \left(n\right)
\end{align*}

\subsection*{Transformaty podstawowych ciągofunkcji}
    \begingroup
    \allowdisplaybreaks
    \begin{align*}
        \mathcal{Z}\left[0\right] &= 0 \\
        \mathcal{Z}\left[\delta\left(n\right)\right] &= 1\\
            &\text{gdzie:} \quad \delta\left(n\right) = \left\{
                \begin{array}{l}
                    1, \quad n = 0 \\
                    0, \quad n \not = 0
                \end{array}
            \right. \\
        \mathcal{Z}\left[\delta \left(n - k\right)\right] &= z^{-k}, \quad n \leq k \\
%
        \mathcal{Z}\left[1\right] &= \frac{z}{z - 1}, \quad \left| z \right| > 1 \\
        \mathcal{Z}\left[n\right] &= \frac{z}{\left(z - 1 \right)^2}\\
        \mathcal{Z}\left[n^2\right] &= \frac{z \left(z + 1\right)}{\left(z - 1 \right)^3}\\
        \mathcal{Z}\left[n^3\right] &= \frac{z \left(z^2 + 4z + 1\right)}{\left(z - 1 \right)^4}\\
%
        \mathcal{Z}\left[a^n\right] &= \frac{z}{z - a}, \quad \left| z \right| > \left| a \right| \\
        \mathcal{Z}\left[\left(-a\right)^n\right] &= \frac{z}{z + a} \\
        \mathcal{Z}\left[na^n\right] &= \frac{az}{\left(z - a\right)^2} \\
        \mathcal{Z}\left[n^2 a^n\right] &= \frac{az\left(z+a\right)}{\left(z - a\right)^3} \\
        \mathcal{Z}\left[\sin \omega n\right] &= \frac{z \sin \omega}{z^2 - 2z \cos \omega + 1} \\
        \mathcal{Z}\left[\cos \omega n\right] &= \frac{z \cos \omega}{z^2 - 2z \cos \omega + 1} \\
    \end{align*}
    \endgroup

\section{Transformata $\mathcal{Z}^{-1}$}

\[
    f \left(n\right) = \frac{1}{2 \pi i} \oint \limits_K F\left(z\right) z^{n - 1}\, dz
\]

\section{Splot ciągów} 

\[
    h \left( n \right) = f \left( n \right) * g \left( n \right) = \sum \limits^n_{i = 0} f \left(i\right) g \left( n - i\right)
\]

\subsection*{Twierdzenie}

\[
    f \left( n \right) * g \left( n \right) \odpowiada F \left(z\right) \cdot G \left( z \right)
\]
\end{multicols}
\end{document}
